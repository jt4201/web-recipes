%
% arara: xelatex
% arara: xelatex
% arara: latexmk: { options: ['-c'] }
% vim: tw=86:
\documentclass[../cookbook]{subfiles}
\graphicspath{{\subfix{../images/}}}
\begin{document}

\renewcommand{\myTitleText}{Orange Ginger Pork Kebab Marinade}
\renewcommand{\myURLText}{https://mylifecookbook.com/orange-ginger-pork-kebabs/}
\renewcommand{\myFilename}{MLC-orange-ginger-pork-kebab}
\renewcommand{\myServings}{6}
\renewcommand{\myAuthor}{My Life Cookbook}
\renewcommand{\myCookTime}{10 minutes}
\renewcommand{\myPrepTime}{10 minutes}
\renewcommand{\myTotalTime}{20 minutes}
\renewcommand{\myPreamble}{
  This orange ginger pork kebab marinade is an easy and delicious summer dinner on the
  grill. The ginger orange marinade adds so much flavor and is low carb. You can grill
  these sweet and spice pork kebabs in about 15 minutes so they are perfect for a busy
  weeknight dinner! Each serving has just 6.4g net carbs.
}

\sbox{\myIngred}{%
  \begin{minipage}[t]{\the\dimexpr0.35\textwidth}
    \textbf{Ingredients}
    \begin{itemize}
      \item 3 lbs pork loin country ribs, cut into cubes
      \item 1 orange, zested
      \item ¼ cup soy sauce
      \item 2 cloves of garlic, crushed
      \item ¼ cup olive oil
      \item 2 tablespoons orange marmalade
      \item 1 inch ginger root, grated
      \item 2 tablespoons orange juice
    \end{itemize}
  \end{minipage}
}

\sbox{\myMethod}{%
  \begin{minipage}[t]{\the\dimexpr0.6\textwidth-2em}
    \textbf{Method}
    \begin{enumerate}
      \item Whisk everything but the pork in a medium sized bowl.
      \item Place your pork cubes in a large ziplock bag and then pour the marinade
        over it.
      \item Mix the meat around in the bag with your hand to make sure they are all
        coated.
      \item Place in the refrigerator for 6 hours or over night.
      \item Thread the pork cubes on skewers then place on a hot grill grate. Grill
        for about 4--5 minutes on both sides. Take off grill and let cool then serve.
      \item Store leftover in an airtight containers. Left overs are great on salads
        or can be used in a fried rice dish. 
    \end{enumerate}
  \end{minipage}
}

\sbox{\myTitle}{%
  \begin{minipage}[t]{0.6\textwidth}\raggedright%
    \ifthenelse{\equal{\myTitleText}{}} {}%
    {\textbf{\LARGE\bfseries \myTitleText} \vspace{2ex} \\ }
    \ifthenelse{\equal{\myPrepTime}{} } {}%
    {\cmdSubtitle{PREP~TIME}{\myPrepTime}\quad}
    \ifthenelse{\equal{\myCookTime}{} } {}%
    {\cmdSubtitle{COOK~TIME}{\myCookTime} \\ }
    \ifthenelse{\equal{\myTotalTime}{} } {}%
    {\cmdSubtitle{TOTAL~TIME}{\myTotalTime}\quad}
    \ifthenelse{\equal{\myServings}{} } {}%
    {\cmdSubtitle{SERVINGS}{\myServings} \\ }
    \ifthenelse{\equal{\myAuthor}{} } {}%
    {\cmdSubtitle{AUTHOR}{\myAuthor} \\ }
    \ifthenelse{\equal{\myURLText}{} } {}%
    {\cmdSubtitle{SOURCE}{\href{\myURLText}{\myTitleText} \vspace{1ex}} \\ }
    \ifthenelse{\equal{\myPreamble}{}} {}%
    {\myPreamble}
  \end{minipage}
}

\sbox{\myImage}{%
  \begin{minipage}[t]{0.35\textwidth}
    \adjustbox{valign=t}{\includegraphics[width=\textwidth]{\myFilename}}
  \end{minipage}
}

\framebox{%
  \cmdRecipe{recipelist}{Mains!Kebabs}{\myTitleText}
  \cmdRecipe{authorlist}{\myAuthor}{\myTitleText}
\begin{tabular}{@{}llllll@{}}
  \multicolumn{4}{@{}l@{}}{\usebox{\myTitle}} &
  \multicolumn{2}{@{}l@{}}{\usebox{\myImage}} \\
  \multicolumn{6}{@{}l@{}}{} \rule{0.95\linewidth}{0.4pt}\vspace{2ex} \\ 
  \multicolumn{2}{@{}l@{}}{\usebox{\myIngred}} &
  \multicolumn{4}{@{}l@{}}{\usebox{\myMethod}} \\
\end{tabular}
}

\end{document}
