% arara: xelatex
\documentclass[../cookbook]{subfiles}
\begin{document}

\renewcommand{\myAuthor}{Cook Report}
\renewcommand{\myTitletext}{Potato Curry with Creamy Tomato Sauce}
\renewcommand{\myUrltext}{https://thecookreport.co.uk/potato-curry/}
\renewcommand{\myServingstext}{4}
\renewcommand{\mycourse}{Mains}

\hypersetup{
  pdftitle = {\myTitletext},
  pdfauthor = {\myAuthor}
}

\sbox{\Ingredients}{%
  \begin{minipage}[t]{\the\dimexpr 0.4\textwidth - 2em}
    \par\mySubtitle{Ingredients}
    \par\begin{itemize}
      \item 3 tablespoon Olive Oil
      \item 1 Red Onion
      \item 1 teaspoon Ginger grated
      \item 4 cloves Garlic crushed
      \item 1 tablespoon Garam Masala
      \item 1 Bay Leaf
      \item Pinch Chilli Flakes
      \item 2 tablespoon Honey
      \item 2 400g tins Chopped Tomatoes
      \item 1 400g tin Coconut Milk
      \item 800 g Baby Potatoes big ones halved
      \item 4 tablespoon Greek Yoghurt
      \item Coriander to serve
      \item Salt
    \end{itemize}
  \end{minipage}
}

\sbox{\Instructions}{%
  \begin{minipage}[t]{\the\dimexpr 0.6\textwidth - 2em}
    \par\mySubtitle{Instructions}
    \par\begin{enumerate}
      \item Heat the olive oil in a large pot over a medium heat and add the onion. Cook
        for a couple of minutes then add the garlic and ginger, sprinkle with salt and
        cook for another few minutes. Add the garam masala, bay leaf and chilli flakes,
        stir to coat then add the honey and cook for another minute.
      \item Pour in the tomatoes and bring to a simmer, cook for 10 minutes until
        thickened slightly. Add the coconut milk and the potatoes and cook for another
        20-25 minutes until the potatoes are cooked through. Add salt to taste.
      \item Mix the yoghurt with 1-2 tablespoon water and a pinch of salt. Serve the curry
        with rice, naan and a drizzle of yoghurt sauce.
    \end{enumerate}
  \end{minipage}
}

\framebox{
  \cmdRecipe{recipelist}{\mycourse}{\myTitletext}
  \begin{tabular}{ll}
    \multicolumn{2}{l}{\LARGE\bfseries \myTitletext (\myAuthor)} \\ \addlinespace[1mm]
    \multicolumn{2}{l}{Servings: \myServings{\myServingstext}} \\
  \multicolumn{2}{l}{\myUrl{\myUrltext}{\myTitletext}} \\ \addlinespace[3mm]
  \usebox{\Ingredients} & \usebox{\Instructions} \\ \addlinespace[3mm]
\end{tabular}
}

\end{document}
