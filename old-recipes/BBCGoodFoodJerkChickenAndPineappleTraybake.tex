% vim: ft=tex:
\documentclass[../web-recipes.tex]{subfiles}
\begin{document}

\pagestyle{empty}

\renewcommand{\mytitle}{Jerk chicken \& pineapple traybake (BBC Good Food)}
\renewcommand{\myurl}{https://www.bbcgoodfood.com/recipes/jerk-chicken-pineapple-traybake}

\myRecipe{Mains}{\mytitle{}}

\sbox{\ingredients}{%
  \begin{minipage}[t]{\the\dimexpr 0.35\linewidth - 1em\relax}
    {\sffamily\bfseries Ingredients}\vspace{0.5ex}
    \begin{itemize}
      \item 6 chicken thighs, skin on and bone in
      \item 90g jerk paste
      \item 3 limes, 1 zested and juiced, 2 cut into wedges, to serve
      \item 1 pineapple, peeled, cored and chopped into chunks
      \item 2 sweet potatoes, peeled and chopped into chunks
      \item 1 red pepper, sliced
      \item 300g wholegrain rice
      \item 400g can black beans
      \item 1 tbsp coconut oil
      \item few thyme sprigs, leaves picked
      \item 3 Little Gem lettuces, sliced
    \end{itemize}
  \end{minipage}
}

\sbox{\method}{%
  \begin{minipage}[t]{\the\dimexpr 0.6\linewidth - 1em\relax}
    {\sffamily\bfseries Method}\vspace{0.5ex}
    \begin{enumerate}
      \item Heat oven to 180C/160C fan/gas 4. Cut the thighs in half lengthways down one side of the
        bone. Put them in a large roasting tin. Mix the jerk paste with the lime zest and juice and
        pour over the chicken. Tuck the pineapple pieces, sweet potatoes and peppers in and around
        the chicken. Season, cover with foil and roast for 30 mins.
      \item Increase the oven temperature to 220C/200C fan/gas 7, remove the foil, baste the chicken
        and stir the veg in the cooking juices. Return to the oven for 20-25 mins.
      \item Meanwhile, cook the rice according to pack instructions, then stir in the beans, coconut
        oil, thyme leaves and seasoning. Cook for 2-3 mins to warm the beans through and melt the
        coconut oil. Serve 3 pieces of chicken per person with the vegetables, and the lettuce and
        rice on the side.
    \end{enumerate}
  \end{minipage}
}


  \begin{tabular}{l}
    \usebox{\ingredients}\hspace{1em}\usebox{\method}\vspace{3ex}\\
    \multicolumn{1}{c}{\small\ttfamily \url{\myurl}} \\
  \end{tabular}

\end{document}
