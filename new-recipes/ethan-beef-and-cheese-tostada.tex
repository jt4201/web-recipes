% nop
% arara: xelatex
% arara: xelatex
% arara: latexmk: { options: ['-c'] }
% vim: tw=86:
\documentclass[../cookbook]{subfiles}
\graphicspath{{\subfix{../images/}}}
\begin{document}

\renewcommand{\myTitleText}{Bean \& Cheese Tostada}
\renewcommand{\myURLText}{https://www.ethanchlebowski.com/cooking-techniques-recipes/bean-amp-cheese-tostada-optional-breakfast-variation}
\renewcommand{\myFilename}{ethan-beef-and-cheese-tostada.jpg}
\renewcommand{\myServings}{1 tostada}
\renewcommand{\myAuthor}{Ethan Chlebowski}
\renewcommand{\myCookTime}{}
\renewcommand{\myPrepTime}{}
\renewcommand{\myTotalTime}{}
\renewcommand{\myIndexEntry}{Mains}
\renewcommand{\myPreamble}{
}

\hypersetup{%
  pdfauthor={\myAuthor},
  pdftitle={\myTitleText}
}

\sbox{\myIngred}{%
  \begin{minipage}[t]{\the\dimexpr0.35\textwidth}
    \textbf{Ingredients}
    \begin{jitem}
      \item Tostada Components
        \begin{jsubitem}
        \item Tostada
        \item Refried Beans
        \item Crumbled queso fresco or Monterrey Jack cheese
        \item Crema or sour cream
        \end{jsubitem}
      \item Optional Breakfast Additions
        \begin{jsubitem}
        \item Fried egg
        \item Hot sauce
        \item Cilantro
        \end{jsubitem}
      \item Optional Lunch Additions
        \begin{jsubitem}
        \item Thinly shredded lettuce
        \item Diced tomatoes
        \item Pickled onions
        \end{jsubitem}
      \item Refried Beans  (makes enough for 3-4 tostadas)
        \begin{jsubitem}
        \item 1 can (14 oz) of pinto beans
        \item Drizzle of oil
        \item 2 cloves of garlic, thinly sliced
        \item 1 g (1/2 tsp) of whole cumin seeds
        \item Sprinkle of smoked paprika
        \item Salt \& lime juice to taste
        \end{jsubitem}
      \end{jitem}
    \end{minipage}
  }

  \sbox{\myMethod}{%
    \begin{minipage}[t]{\the\dimexpr0.6\textwidth-2em}
      \textbf{Method}
      \begin{jitem}
      \item Start the refried beans
        \begin{jnum}[label=,labelsep=0pt,itemindent=-0.5em,leftmargin=1em,]
        \item To start, set a pan over medium-high heat and add a drizzle of oil to the
          pan
        \item Next, add the sliced garlic and cumin seeds. Let these toast for 60
          seconds to flavor the oil.
        \item Then, right before adding the beans, sprinkle in some smoked paprika.
          Follow this by adding one can of pinto beans, including the liquid.
        \item Bring this mixture to a simmer and let it cook for about 5 minutes.
        \end{jnum}
      \item Finish \& adjust the beans
        \begin{jnum}[label=,labelsep=0pt,itemindent=-0.5em,leftmargin=1em,]
        \item   Using a potato masher or fork, mash the beans until they are mostly broken down, but you can make this smoother or chunkier depending on your preference.
        \item   If you want it thicker, let it simmer for an additional couple of minutes so more moisture evaporates.
        \item   Once you’ve reached your desired texture, shut off the heat and taste. Adjust with salt and a squeeze of lime juice as needed.
        \end{jnum}
      \item Gather ingredients
        \begin{jnum}[label=,labelsep=0pt,itemindent=-0.5em,leftmargin=1em,]
        \item   Gather the tostadas and any toppings you want.
        \item   If making the breakfast version, fry up an egg per tostada at this time.
        \end{jnum}
      \item Assemble \& serve
        \begin{jnum}[label=,labelsep=0pt,itemindent=-0.5em,leftmargin=1em,]
        \item   To a tostada, add your cheese first and follow that with the hot beans over the top so it starts to soften and melt a little bit.
        \item   Once the beans are added, add a drizzle of crema or sour cream. Then finish it:
        \item   For the breakfast version, lay the fried egg on top, and finish with any other toppings.
        \item   For the lunch version, top with shredded lettuce, red onion, diced tomatoes, or your toppings of choice.
        \end{jnum}
      \end{jitem}
    \end{minipage}
  }

\newcommand{\cmdSubText}[2]{%
  \ifthenelse{\equal{#2}{}} {} {\cmdSubtitle{#1}{#2}}
}

\sbox{\myTitle}{%
  \begin{minipage}[t]{0.6\textwidth}\raggedright%
    \textbf{\LARGE\bfseries \myTitleText} \vspace{2ex} \\
    \cmdSubText{PREP~TIME}{\myPrepTime}\quad
    \cmdSubText{COOK~TIME}{\myCookTime}\quad
    \cmdSubText{TOTAL~TIME}{\myTotalTime}\quad \\
    \cmdSubText{SERVINGS}{\myServings}\vspace{0.5ex} \\
    \cmdSubText{AUTHOR}{\myAuthor}\\
    \cmdSubText{SOURCE}{\href{\myURLText}{\myTitleText} \vspace{1ex} \\}
    \myPreamble{}
  \end{minipage}
}

\sbox{\myImage}{%
  \begin{minipage}[t]{0.35\textwidth}
    \ifthenelse{\equal{\myFilename}{}}{}{\adjustbox{valign=t}{\includegraphics[width=\textwidth]{\myFilename}}}
  \end{minipage}
}

\framebox{%
  \cmdRecipe{recipelist}{\myIndexEntry}{\myTitleText}
  \cmdRecipe{authorlist}{\myAuthor}{\myTitleText}
\begin{tabular}{@{}llllll@{}}
  \multicolumn{4}{@{}l@{}}{\usebox{\myTitle}} &
  \multicolumn{2}{@{}l@{}}{\usebox{\myImage}} \\
  \multicolumn{6}{@{}l@{}}{} \rule{0.95\linewidth}{0.4pt}\vspace{2ex} \\ 
  \multicolumn{2}{@{}l}{\usebox{\myIngred}} &
  \multicolumn{4}{@{}l}{\usebox{\myMethod}} \\
\end{tabular}
}

\end{document}
