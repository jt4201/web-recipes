% arara: xelatex
\documentclass[web-recipes.tex]{subfiles}
\begin{document}

\renewcommand{\myAuthor}{Taste of Home}
\renewcommand{\myTitletext}{Pico de Gallo Black Bean Soup}
\renewcommand{\myUrltext}{https://www.tasteofhome.com/recipes/pico-de-gallo-black-bean-soup/}
\renewcommand{\myServingstext}{6}
\renewcommand{\mycourse}{Soups}

\hypersetup{
  pdftitle = {\myTitletext},
  pdfauthor = {\myAuthor}
}

\sbox{\Ingredients}{%
  \begin{minipage}[t]{\the\dimexpr 0.4\textwidth - 2em}
    \par\mySubtitle{Ingredients}
    \par\begin{itemize}
      \item 4 cans (15 ounces each) black beans, rinsed and drained
      \item 2 cups vegetable broth
      \item 2 cups pico de gallo
      \item \nicefrac{1}{2} cup water
      \item 2 teaspoons ground cumin
      \item \textbf{Optional toppings}
        \begin{itemize}
          \item Chopped fresh cilantro and additional pico de gallo
        \end{itemize}
    \end{itemize}
  \end{minipage}
}

\sbox{\Instructions}{%
  \begin{minipage}[t]{\the\dimexpr 0.6\textwidth - 2em}
    \par\mySubtitle{Instructions}
    \par\begin{enumerate}
      \item In a Dutch oven, combine the first 5 ingredients; bring to a boil over medium
        heat, stirring occasionally. Reduce heat; simmer, uncovered, until vegetables in
        pico de gallo are softened, 5-7 minutes, stirring occasionally.
      \item Puree soup using an immersion blender, or cool soup slightly and puree in
        batches in a blender. Return to pan and heat through. Serve with toppings as
        desired. 
    \end{enumerate}
  \end{minipage}
}

\begin{tabular}{ll}
  \multicolumn{2}{l}{\myTitle{\mycourse}{\myTitletext\quad (\myAuthor)}} \\ \addlinespace[1mm]
	\multicolumn{2}{l}{\myServings{\myServingstext}} \\
  \multicolumn{2}{l}{\myUrl{\myUrltext}{\myTitletext}} \\ \addlinespace[3mm]
  \usebox{\Ingredients} & \usebox{\Instructions} \\ \addlinespace[3mm]
\end{tabular}

\end{document}
