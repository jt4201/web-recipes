\documentclass[../web-recipes.tex]{subfiles}
\begin{document}

\renewcommand{\mytitle}{Black Bean and Rice Bake (the Candid Cooks)}
    \renewcommand{\myurl}{https://www.thecandidcooks.com/black-bean-and-rice-bake/}
\myRecipe{Mains}{\mytitle{}}

% \addcontentsline{toc}{subsubsection}{\mytitle}
%       \subsubsection*{\mytitle}
      \begin{tabular}{l}
        \begin{minipage}[t]{0.35\textwidth}
          {\sffamily\bfseries Ingredients}\vspace{0.5ex}
              \begin{itemize}
                \item 1 cup jasmine rice dry
                \item 2 cups water
                \item 2 tbsp vegetable or avocado oil divided
                \item 1 tsp minced garlic
                \item 1 medium jalapeno pepper diced
                \item 1 medium poblano (pasilla) pepper sliced thin and into 2-inch long strips
                \item 1 medium roma tomato diced
                \item 15 oz canned black beans drained and rinsed
                \item \nicefrac{1}{4} tsp ground red cayenne pepper
                \item \nicefrac{1}{4} tsp paprika
                \item \nicefrac{1}{4} tsp cumin
                \item 1 pinch salt
                \item \nicefrac{1}{3} cup shredded Mexican-blend cheese
                \item 1 small avocado sliced
              \end{itemize}
        \end{minipage}
        \qquad
        \begin{minipage}[t]{0.55\textwidth}
          {\sffamily\bfseries Method}\vspace{0.5ex}
          \begin{enumerate}
            \item First, make the rice. Bring water to a boil in a small
              saucepan and stir in jasmine rice. Cover and cook over low heat
              until the water has evaporated and rice is fluffy, about 20
              minutes.
            \item While the rice is cooking, prepare the black beans and
              veggies. Heat 1 tbsp oil in a large cast iron skillet over medium
              heat until shimmering. Add garlic, jalapeno, and pasilla/poblano
              peppers and sauté until peppers are slightly softened, about 2 to
              3 minutes.
            \item Next, add the tomatoes and black beans. Season with cayenne
              pepper, paprika, cumin, and salt and cook until beans are
              slightly softened and tomato juices have started to come out,
              about 3 to 5 minutes. Then transfer black beans and veggies to a
              bowl and set aside.
            \item Preheat the oven to 350 degrees F. Return the cast iron pan
              to medium-high heat and add the remaining 1 tbsp oil, carefully
              swirling it around to coat the inside of the pan.
            \item Add the cooked rice to the cast iron pan. Spread it out into
              an even layer and gently press down on it with your spatula. Let
              it crisp until lightly golden on the bottom, about 3 to 5
              minutes, then stir and press down into an even layer again. Let
              it crisp one more time, about 3 to 5 minutes, until golden. Then
              stir again, and press down into the bottom of the pan and around
              the edges to create a wide, shallow well for the beans and
              veggies.
            \item Remove the cast iron pan from heat. Sprinkle shredded cheese
              evenly over the rice, then top with the black beans and veggies.
              Bake in the oven for 10 minutes to let the flavors combine.
            \item Serve hot, topped with avocado slices and an optional garnish
              of chopped cilantro.
          \end{enumerate}
        \end{minipage} \vspace{3ex}\\
        \multicolumn{1}{c}{\small\ttfamily \url{\myurl}} \\
      \end{tabular}
    \end{document}
