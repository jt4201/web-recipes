\documentclass[../web-recipes.tex]{subfiles}
\begin{document}

\renewcommand{\mytitle}{Black Bean and Corn Salsa (Emily Enchanted)}
\renewcommand{\myurl}{https://www.emilyenchanted.com/black-bean-and-corn-salsa/}
\myRecipe{Sauces!Salsa}{\mytitle{}}

% \addcontentsline{toc}{subsubsection}{\mytitle}
%       \subsubsection*{\mytitle}
      \begin{tabular}{l}
        \begin{minipage}[t]{0.35\textwidth}
          {\sffamily\bfseries Ingredients}\vspace{0.5ex}
              \begin{itemize}
                \item 15 oz Can Black Beans, drained and rinsed
                \item 15 oz Can Corn, drained and rinsed
                \item \nicefrac{1}{4} Medium Red Onion, chopped
                \item 1 Red Bell Pepper, chopped
                \item 3 Tablespoons Scallions, chopped
                \item \nicefrac{1}{4} cup Cilantro, chopped
                \item \nicefrac{1}{8} cup Extra Virgin Olive Oil
                \item 2 Limes, juiced
                \item \nicefrac{1}{2} teaspoon Dried Oregano
                \item \nicefrac{1}{2} teaspoon Smoked Paprika
                \item Salt and Pepper, to taste
              \end{itemize}
        \end{minipage}
        \qquad
        \begin{minipage}[t]{0.55\textwidth}
          {\sffamily\bfseries Method}\vspace{0.5ex}
          \begin{enumerate}
            \item Place black beans, corn, red onion, red bell pepper,
              scallions and cilantro in a large bowl. Mix thoroughly, set
              aside.
            \item Add extra virgin olive oil, lime juice, dried oregano, smoked
              paprika and salt and pepper to a medium bowl. Mix well until all
              ingredients are well incorporated.
            \item Pour liquid mixture over salsa and mix thoroughly so the
              salsa is evenly coated. If desired, add additional salt and
              pepper, to taste.
            \item Cover and refrigerate for at least one hour before serving.
          \end{enumerate}
        \end{minipage} \vspace{3ex}\\
        \multicolumn{1}{c}{\small\ttfamily \url{\myurl}} \\
      \end{tabular}
    \end{document}
