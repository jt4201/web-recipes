% arara: xelatex
% arara: xelatex
% arara: latexmk: { options: ['-c'] }
% vim: tw=86:
\documentclass[../cookbook]{subfiles}
\graphicspath{{\subfix{../images/}}}
\begin{document}

\renewcommand{\myTitleText}{campfire Apple Crisp}
\renewcommand{\myURLText}{https://www.freshoffthegrid.com/no-bake-apple-bourbon-cobbler/}
\renewcommand{\myFilename}{fotg-campfire-apple-crisp.jpg}
\renewcommand{\myServings}{2 servings}
\renewcommand{\myAuthor}{Fresh Off The Grid}
\renewcommand{\myCookTime}{15 minutes}
\renewcommand{\myPrepTime}{5 minutes}
\renewcommand{\myTotalTime}{20 minutes}
\renewcommand{\myPreamble}{
  A quick and easy fall-inspired dessert, this vegan campfire apple crisp is
  guaranteed to leave you feeling warm and cozy at the end of the night.
}

\sbox{\myIngred}{
  \begin{minipage}[t]{\the\dimexpr 0.35\textwidth}\raggedright
    \textbf{Ingredients}
    \begin{itemize}
      \item 2-3 apples sliced ¼” thin
      \item 1 tablespoon oil or butter
      \item 1 teaspoon ground cinnamon
      \item ½ teaspoon ground nutmeg
      \item ½ teaspoon ground cloves
      \item 2 tablespoons bourbon
      \item ¼ cup brown sugar
      \item 1 cup granola
    \end{itemize}
  \end{minipage}
}

\sbox{\myMethod}{
  \begin{minipage}[t]{\the\dimexpr 0.55\textwidth - 2em}\raggedright
    \textbf{Method}
    \begin{enumerate}
      \item In an 8" or 10” cast iron skillet, over medium heat, saute the sliced
        apples in a bit of oil or butter until beginning to soften, about 5 minutes.
      \item Add the spices, bourbon, and sugar, and stir to combine. Continue cooking
        until the sauce thickens and the apples are tender, 5-10 minutes.
      \item Remove from the heat and sprinkle the granola evenly across the top. Enjoy
        straight from the skillet or serve in individual bowls with a dollop of
        whipped cream.
    \end{enumerate}
  \end{minipage}
}

\begin{tabular}{ll}
  \begin{minipage}[t]{0.6\textwidth}\raggedright
  \cmdRecipe{recipelist}{Dessert}{\myTitleText}
  \cmdRecipe{authorlist}{\myAuthor}{\myTitleText}
    \ifthenelse{\equal{\myTitleText}{}} {}%
    {\textbf{\LARGE\bfseries \myTitleText} \vspace{2ex} \\ }
    \ifthenelse{ \equal{\myPrepTime}{} } {}%
      {\cmdSubtitle{PREP~TIME}{\myPrepTime}\quad}
      \ifthenelse{ \equal{\myCookTime}{} } {}%
      {\cmdSubtitle{COOK~TIME}{\myCookTime} \\ }
      \ifthenelse{ \equal{\myTotalTime}{} } {}%
    {\cmdSubtitle{TOTAL~TIME}{\myTotalTime}\quad}
      \ifthenelse{ \equal{\myServings}{} } {}%
      {\cmdSubtitle{SERVINGS}{\myServings} \\ }
      \ifthenelse{ \equal{\myAuthor}{} } {}%
      {\cmdSubtitle{AUTHOR}{\myAuthor} \\ }
      \ifthenelse{ \equal{\myURLText}{} } {}%
      {\cmdSubtitle{SOURCE}{\href{\myURLText}{\myTitleText} \vspace{1ex}} \\ }
    \ifthenelse{ \equal{\myPreamble}{}} {}%
      {\myPreamble}
\end{minipage} &
\begin{minipage}[t]{0.35\textwidth}
  \adjustbox{valign=t}{\includegraphics[width=\textwidth]{\myFilename}}
\end{minipage} \\
\end{tabular}\vspace{5mm}

\par\noindent\rule{0.95\textwidth}{0.4pt}\vspace{5mm}

\begin{tabular}{ll}
  \usebox{\myIngred} & \usebox{\myMethod} \\
\end{tabular}

\end{document}
