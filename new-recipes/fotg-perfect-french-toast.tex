% arara: xelatex
% arara: xelatex
% vim: tw=86:
\documentclass{newRecipe}
\begin{document}

% FIXME:
\renewcommand{\myTitleText}{Perfect French Toast}
\renewcommand{\myURLText}{https://www.freshoffthegrid.com/camping-french-toast/}
\renewcommand{\myFilename}{fotg-perfect-french-toast.png}
\renewcommand{\myServings}{8 slices}
\renewcommand{\myAuthor}{Fresh Off The Grid}
\renewcommand{\myCookTime}{20 minutes}
\renewcommand{\myPrepTime}{10 minutes}
\renewcommand{\myTotalTime}{30 minutes}
\renewcommand{\myPreamble}{
  This recipe for classic French Toast is one you'll return to again and again!
}

\sbox{\myIngred}{
  \begin{minipage}[t]{\the\dimexpr 0.35\textwidth}\raggedright
    \textbf{Ingredients}
    \begin{itemize}
      \item ½ lb loaf bread
      \item 3 eggs
      \item 1 cup milk
      \item 2 tablespoons sugar plus more for sprinkling
      \item 1 teaspoon cinnamon
      \item 1 teaspoon vanilla extract optional
      \item ¼ teaspoon nutmeg optional
      \item 4 tablespoons butter for pan
      \item Maple syrup \& berries to top
    \end{itemize}
  \end{minipage}
}

\sbox{\myMethod}{
  \begin{minipage}[t]{\the\dimexpr 0.55\textwidth - 2em}\raggedright
    \textbf{Method}
    \begin{enumerate}
      \item Cut the bread into 3/4" - 1” thick slices.
      \item Beat the eggs first in a bowl large enough to accommodate a slice of the
        bread. Then add milk, cinnamon, nutmeg, vanilla, and sugar together, until
        thoroughly mixed.
      \item Heat 4 tablespoon butter in a skillet over medium heat.
      \item Dip a slice of bread in the egg and milk mixture and let it soak for about
        10 seconds on each side. Let the excess drip off, sprinkle each side with
        additional sugar, and then fry it in the skillet until golden and crispy on
        each side, about 3 minutes per side.
      \item Repeat with the rest of the bread, adding more butter to the skillet as
        needed.
      \item Serve with maple syrup, fresh fruit, and a cup of hot coffee. Enjoy!
    \end{enumerate}
  \end{minipage}
}

\begin{tabular}{ll}
  \begin{minipage}[t]{0.6\textwidth}\raggedright
    \ifthenelse{\equal{\myTitleText}{}} {}%
    {\textbf{\LARGE\bfseries \myTitleText} \vspace{2ex} \\ }
    \ifthenelse{ \equal{\myPrepTime}{} } {}%
      {\cmdSubtitle{PREP~TIME}{\myPrepTime}\quad}
      \ifthenelse{ \equal{\myCookTime}{} } {}%
      {\cmdSubtitle{COOK~TIME}{\myCookTime} \\ }
      \ifthenelse{ \equal{\myTotalTime}{} } {}%
    {\cmdSubtitle{TOTAL~TIME}{\myTotalTime}\quad}
      \ifthenelse{ \equal{\myServings}{} } {}%
      {\cmdSubtitle{SERVINGS}{\myServings} \\ }
      \ifthenelse{ \equal{\myAuthor}{} } {}%
      {\cmdSubtitle{AUTHOR}{\myAuthor} \\ }
      \ifthenelse{ \equal{\myURLText}{} } {}%
      {\cmdSubtitle{SOURCE}{\href{\myURLText}{\myTitleText} \vspace{1ex}} \\ }
    \ifthenelse{ \equal{\myPreamble}{}} {}%
      {\myPreamble}
\end{minipage} &
\begin{minipage}[t]{0.35\textwidth}
  \adjustbox{valign=t}{\includegraphics[width=\textwidth]{\myFilename}}
\end{minipage} \\
\end{tabular}\vspace{5mm}

\par\noindent\rule{0.95\textwidth}{0.4pt}\vspace{5mm}

\begin{tabular}{ll}
  \usebox{\myIngred} & \usebox{\myMethod} \\
\end{tabular}

\end{document}
