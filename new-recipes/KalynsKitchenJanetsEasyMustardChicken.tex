% vim: ft=tex:
\documentclass[../cookbook]{subfiles}
\begin{document}

\pagestyle{empty}

\renewcommand{\mytitle}{Janet’s Easy Mustard Chicken (Kalyn's Kitchen)}
\renewcommand{\myurl}{https://kalynskitchen.com/janets-mustard-chickenan-old-favorite/}

\myRecipe{Mains}{\mytitle{}}

\sbox{\ingredients}{%
  \begin{minipage}[t]{\the\dimexpr 0.35\linewidth - 1em\relax}
    {\sffamily\bfseries Ingredients}\vspace{0.5ex}
    \begin{description}
      \item[Chicken] \hfill
        \begin{itemize}
          \item 8 boneless, skinless chicken breasts
          \item 1 - 2 T olive oil (will depend on your pan)
          \item 2 tsp. poultry seasoning
        \end{itemize}
      \item[Sauce] \hfill
        \begin{itemize}
          \item \nicefrac{1}{2} cup mayo
          \item \nicefrac{1}{2} cup sour cream
          \item 2 T Dijon mustard
          \item 1 tsp. dry mustard
          \item \nicefrac{1}{2} tsp. ground ginger
          \item 1 \nicefrac{1}{2} cups chicken stock (see notes)
          \item salt and fresh-ground black pepper to taste
          \item \textbf{optional}
            \begin{itemize}
              \item chopped parsley for garnish (optional)
            \end{itemize}
        \end{itemize}
    \end{description}
  \end{minipage}
}

\sbox{\method}{%
  \begin{minipage}[t]{\the\dimexpr 0.6\linewidth - 1em\relax}
    {\sffamily\bfseries Method}\vspace{0.5ex}
    \begin{enumerate}
      \item Trim all visible fat and undesirable parts from chicken breasts, then make
        very small slits crosswise down the length of each breast. (This helps the chicken
        to cook evenly. Be careful not to cut through too far.)
      \item Season chicken with Poultry Seasoning on both sides.
      \item Heat olive oil in large frying pan big enough to hold the chicken in a single
        layer, add chicken and cook about 4 minutes on the first side, until chicken is
        lightly browned.
      \item Turn chicken over and cook 2-3 minutes on the second side, until chicken feels
        almost, but not quite firm to the touch. (It will continue to cook in the sauce,
        so don’t overcook.)
      \item While chicken cooks, put all sauce ingredients in bowl or large measuring cup,
        and whisk to combine.
      \item When chicken is lightly browned, lower heat and pour sauce over.
      \item Simmer 20 minutes, being sure to keep heat at a very low simmer.
      \item After 20 minutes, remove chicken from pan and whisk the sauce a few times to
        be sure it’s smooth.
      \item Season to taste with a little salt and fresh-ground black pepper.
      \item Serve hot, with a little sauce spooned over each piece of chicken, garnishing
        with fresh parsley if desired.
    \end{enumerate}
  \end{minipage}
}


  \begin{tabular}{l}
    \usebox{\ingredients}\hspace{1em}\usebox{\method}\vspace{3ex}\\
    \multicolumn{1}{c}{\small\ttfamily \url{\myurl}} \\
  \end{tabular}

\end{document}
