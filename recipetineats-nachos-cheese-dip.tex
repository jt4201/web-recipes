\documentclass[web-recipes.tex]{subfiles}
\begin{document}

\renewcommand{\mytitle}{Nachos Cheese Dip (RecipeTinEats)}
\renewcommand{\myurl}{https://www.recipetineats.com/nachos-cheese-dip-sauce/}
\begin{mdframed}[nobreak]
  \section{\mytitle}
  \begin{tabular}{l}
    \begin{minipage}[t]{0.35\textwidth}
      {\sffamily\bfseries Ingredients}\vspace{0.5ex}
      \begin{description}
        \item[part one] \hfill
          \begin{itemize}
            \item 2 1/2 cups / 225g / 8oz shredded cheese (Cheddar, Monterey Jack, Colby, etc)
            \item 1 tbsp cornstarch
            \item 1 can evaporated milk (375g / 12oz)
            \item 2 tbsp finely chopped jalapeño peppers (fresh or canned)
            \item 1 tbsp hot sauce
            \item 1 tsp salt
            \item \textbf{optional}
              \begin{itemize}
                \item 1/2 tsp onion powder
                \item 1/2 tsp garlic powder
              \end{itemize}
          \end{itemize}
      \end{description}
    \end{minipage}
    \qquad
    \begin{minipage}[t]{0.55\textwidth}
      {\sffamily\bfseries Method}\vspace{0.5ex}
      \begin{enumerate}\raggedright\small\sffamily
        \item Toss the cheese and cornstarch together in a saucepan.
        \item Add all other ingredients, cook over medium heat, whisking often
          until the cheese is melted and smooth.
        \item Adjust to your taste if required - with extra hot sauce and salt.
        \item While hot, it will have a sauce like consistency - perfect for
          pouring over nachos.
        \item As it cools, it will thicken and become a dip-like consistency.
          Whisk occasionally to mix in the skin that forms on the top.
        \item Stir in additional evaporated (or even normal) milk to achieve
          the desired consistency if you want to use it as a pourable sauce.
          Note that when you reheat it, it loosens up.
      \end{enumerate}
    \end{minipage} \vspace{3ex}\\
    \multicolumn{1}{c}{\small\ttfamily \myurl} \\
  \end{tabular}
\end{mdframed}
\end{document}
