% arara: xelatex
\documentclass[../cookbook]{subfiles}
\begin{document}

\renewcommand{\myAuthor}{COOKIE + kate}
\renewcommand{\myTitletext}{Basic Vinaigrette}
\renewcommand{\myUrltext}{https://cookieandkate.com/how-to-make-vinaigrette-plus-variations/}
\renewcommand{\myServingstext}{¾ cup}
\renewcommand{\mycourse}{Sauces\&Condiments!Dressing}

\hypersetup{
  pdftitle = {\myTitletext},
  pdfauthor = {\myAuthor}
}

\sbox{\Ingredients}{%
  \begin{minipage}[t]{\the\dimexpr 0.4\textwidth - 2em}
    \par\mySubtitle{Ingredients}
    \par\begin{itemize}
      \item ½ cup extra-virgin olive oil
      \item 3 tablespoons vinegar of choice (balsamic vinegar, red wine vinegar, white
        wine vinegar)
      \item 1 tablespoon Dijon mustard
      \item 1 tablespoon maple syrup or honey
      \item 2 medium cloves garlic, pressed or minced
      \item ¼ teaspoon fine sea salt, to taste
      \item Freshly ground black pepper, to taste
    \end{itemize}
  \end{minipage}
}

\sbox{\Instructions}{%
  \begin{minipage}[t]{\the\dimexpr 0.6\textwidth - 2em}
    \par\mySubtitle{Instructions}
    \par\begin{enumerate}
      \item In a liquid measuring cup or bowl, combine all the ingredients. Stir well
        with a small whisk or a fork until the ingredients are completely mixed together.
      \item Taste, and adjust as necessary. If the mixture is too acidic, thin it out with
        a bit more olive oil or balance the flavors with a little more maple syrup or
        honey. If the mixture is a little blah, add another pinch or two of salt. If it
        doesn’t have enough zing, add vinegar by the teaspoon.
      \item Serve immediately, or cover and refrigerate for future use. Homemade
        vinaigrette keeps well for 7 to 10 days. If your vinaigrette solidifies somewhat
        in the fridge, don’t worry about it—real olive oil tends to do that. Simply let it
        rest at room temperature for 5 to 10 minutes or microwave very briefly (about 20
        seconds) to liquify the olive oil again. Whisk to blend and serve.
    \end{enumerate}
  \end{minipage}
}

\sbox{\tips}{%
  \begin{minipage}[t]{\the\dimexpr 0.8\textwidth}
    \par\mySubtitle{Notes}
    \par\begin{description}
      \item[Balsamic vinegar:] Makes a bold, slightly sweet dressing that is wonderful on green salads with fruit, such as apples, strawberries or peaches.
      \item[Red wine vinegar:] Packs a punch and works well with other bold flavors and bright veggies, like tomatoes, bell peppers, cucumber, cabbage and more (think Greek salads).
      \item[White wine vinegar:] This is a more mellow vinegar, and it’s especially nice with more delicate flavors like cucumber and sweet corn. It’s lovely on just about every green salad out there.
      \item[Greek/Italian variation:] Use red wine vinegar. Add 1 to 2 teaspoons dried oregano and, optionally, a pinch of red pepper flakes.
    \end{description}
  \end{minipage}
}

\framebox{
  \cmdRecipe{recipelist}{\mycourse}{\myTitletext}
  \begin{tabular}{ll}
    \multicolumn{2}{l}{\LARGE\bfseries \myTitletext (\myAuthor)} \\ \addlinespace[1mm]
    \multicolumn{2}{l}{Servings: \myServings{\myServingstext}} \\
  \multicolumn{2}{l}{\myUrl{\myUrltext}{\myTitletext}} \\ \addlinespace[3mm]
  \usebox{\Ingredients} & \usebox{\Instructions} \\ \addlinespace[3mm]
\end{tabular}
}

\end{document}
