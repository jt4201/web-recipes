% vim: ft=tex:
\documentclass[web-recipes.tex]{subfiles}
\begin{document}

\pagestyle{empty}

    \renewcommand{\mytitle}{German Rouladen (Chef John)}
    \renewcommand{\myurl}{https://www.allrecipes.com/recipe/244200/chef-johns-beef-rouladen/}

\sbox{\ingredients}{%
  \begin{minipage}[t]{0.32\textwidth}
    {\sffamily\bfseries Ingredients}\vspace{0.5ex}
    \begin{description}
      \item[rouladen] \hfill
        \begin{itemize}
          \item 2 (6 ounce) (\nicefrac{1}{4} inch thick) slices of beef round
          \item 2 teaspoons kosher salt
          \item 1 teaspoon freshly ground black pepper
          \item 2 tablespoons dijon mustard
          \item 4 strips bacon
          \item paprika, or to taste
          \item \nicefrac{1}{2} onion, sliced into half-rings and separated
          \item 6 dill pickle spears
          \item 1 tablespoon vegetable oil
        \end{itemize}
      \item [gravy] \hfill
        \begin{itemize}
          \item 2 tablespoons butter
          \item \nicefrac{1}{4} cup all-purpose flour
          \item 3 cups beef broth
          \item salt to taste
        \end{itemize}
    \end{description}
  \end{minipage}
}

\sbox{\method}{%
  \begin{minipage}[t]{0.55\textwidth}
    {\sffamily\bfseries Method}\vspace{0.5ex}
    \begin{enumerate}
      \item Place slices of beef on a work surface, Season both sides with kosher
        salt and pepper. Spread one side with mustard. Place bacon strips on the
        mustard and sprinkle with paprika. Arrange onion slices cross-wise on the
        beef. Then evenly space 3 pickle slices across each slice of beef. Keep
        about an inch of the narrowest end of the beef slice free of toppings to
        facilitate rolling it up.
      \item Roll each slice of meat, beginning from the wider of the two short
        ends, working to keep all ingredients inside the roll. Rolls should be nice
        and tight. Secure the rolls (seam side down) with 3 loops of butcher's
        twine, one in the middle and one at each end. Trim excess string.
      \item Heat vegetable oil over medium-high heat in a large saucepan. Cook
        beef, turning and browning well on all sides, about 8 minutes. Remove meat
        from pan. Reduce heat to medium-low. Melt butter in the pan; whisk in the
        flour, cooking for about 1 minute. Pour in cold beef broth and whisk
        vigorously to combine. Raise heat to medium-high and simmer until sauce
        begins to thicken, about 1 minute.
      \item Transfer beef rolls to pan along with accumulated juices. Reduce heat
        to very low. Simmer gently, covered, turning rolls every 20 minutes or so,
        until beef is tender, about 1 \nicefrac{1}{2} hours. When the tip of a
        sharp knife can easily be inserted into the beef roll, the meat is done.
      \item Transfer meat to dish to allow it to rest. Raise heat to high and bring
        cooking liquid to a simmer to thicken slightly to make a gravy, about 1
        minute. Serve rouladen with gravy.
    \end{enumerate}
  \end{minipage}
}

\myRecipe{Mains!Rouladen}{\mytitle{}}

% \textbf{\Large \mytitle{}}
% \mainlist{main}{\mytitle}
% \addcontentsline{toc}{subsubsection}{\mytitle}

  \begin{tabular}{l}
    \usebox{\ingredients}\quad\usebox{\method}\vspace{3ex}\\
    \multicolumn{1}{c}{\small\ttfamily \url{\myurl}} \\
  \end{tabular}
\end{document}
