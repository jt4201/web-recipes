% arara: xelatex: { options : "--output-directory=pdf" }
\documentclass[web-recipes.tex]{subfiles}
\usepackage{multicol}
\begin{document}

\renewcommand{\myAuthor}{Food}
\renewcommand{\myTitletext}{Vinaigrette Dressing}
\renewcommand{\myUrltext}{https://www.food.com/recipe/basic-vinaigrette-dressing-with-8-variations-213989}
\renewcommand{\myServingstext}{¼ cup}
\renewcommand{\mycourse}{Sauces!Dressing}

\hypersetup{
  pdftitle = {\myTitletext},
  pdfauthor = {\myAuthor}
}

\sbox{\Ingredients}{%
  \begin{minipage}[t]{\the\dimexpr 0.9\textwidth - 2em}
    \begin{multicols}{3}
    \par\begin{description}
      \item[Basic Vinaigrette] \hfill
        \begin{itemize}
          \item 3 tablespoons oil (I prefer extra-virgin olive oil) 
          \item 2 tablespoons vinegar (white, cider, wine, ..., not balsamic) 
          \item salt 
          \item black pepper (I prefer fresh-ground)
        \end{itemize}
      \item[Italian Vinaigrette] \hfill
        \begin{itemize}
          \item basic vinaigrette (use EVOO and red wine vinegar) 
          \item ½ teaspoon minced garlic 
          \item ½ teaspoon italian seasoning 
          \item 1 pinch crushed red pepper flakes (optional)
        \end{itemize}
      \item[Lighter Bacon Dressing] \hfill
        \begin{itemize}
          \item basic vinaigrette (use canola or corn oil and cider vinegar) 
          \item 1 tablespoon crumbled bacon 
          \item ½ tablespoon finely minced onion 
          \item 1 pinch celery seed (optional) 
          \item ¼ teaspoon prepared mustard (optional) 
          \item 1 -3 teaspoon brown sugar or 1 -3 teaspoon another sweetener, to taste
        \end{itemize}
      \item[Mustard Dressing] \hfill
        \begin{itemize}
          \item italian vinaigrette 
          \item 1 teaspoon prepared mustard (I recommend Dijon or spicy brown)
          \item 1 -3 teaspoon honey (optional) or 1 -3 teaspoon another sweetener, to taste (optional)
        \end{itemize}
      \item[Basic Creamy Vinaigrette Dressing] \hfill
        \begin{itemize}
          \item basic vinaigrette
          \item 2 -3 tablespoons sour cream or 2 -3 tablespoons plain yogurt
        \end{itemize}
      \item[Parmesan-Pepper Dressing] \hfill
        \begin{itemize}
          \item basic creamy vinaigrette dressing 
          \item 1 tablespoon grated parmesan cheese 
          \item ⅛ teaspoon fresh ground black pepper (to taste)
        \end{itemize}
      \item[Creamy Garlic Dressing] \hfill
        \begin{itemize}
          \item basic creamy vinaigrette dressing 
          \item 1 garlic clove, put through press 
          \item fresh ground black pepper 
          \item 1 pinch italian seasoning (optional)
        \end{itemize}
      \item[Lemon Dressing] \hfill
        \begin{itemize}
          \item 3 tablespoons olive oil (I prefer extra-virgin) 
          \item 3 tablespoons lemon juice 
          \item ½ teaspoon oregano 
          \item ½ teaspoon minced garlic
        \end{itemize}
      \item[Balsamic Vinaigrette] \hfill
        \begin{itemize}
          \item 3 tablespoons oil (I prefer extra-virgin olive oil) 
          \item 1 tablespoon balsamic vinegar 
          \item ½ teaspoon minced garlic 
          \item 1 pinch italian seasoning (optional) 
        \end{itemize}
    \end{description}
  \end{multicols}
  \end{minipage}
}

\sbox{\Instructions}{%
  \begin{minipage}[t]{\the\dimexpr 0.6\textwidth - 2em}
    \par\mySubtitle{Instructions}
    \par\begin{enumerate}
      \item Shake all ingredients for your chosen variation together in a tightly-lidded
        container OR whisk together in a small bowl.
      \item Let stand 10 minutes to rehydrate dried herbs and blend flavors.
      \item Shake again then dress salad as desired.
    \end{enumerate}
  \end{minipage}
}

\begin{tabular}{l}
  \myTitle{\mycourse}{\myTitletext\ (\myAuthor)} \\ \addlinespace[1mm]
	\myServings{\myServingstext} \\
  \myUrl{\myUrltext}{\myTitletext} \\ \addlinespace[3mm]
  \usebox{\Ingredients} \\ \addlinespace[3mm]
  \usebox{\Instructions} \\ \addlinespace[3mm]
\end{tabular}

\end{document}
