\documentclass[web-recipes.tex]{subfiles}
\begin{document}

\renewcommand{\mytitle}{Indian Butter Chicken Burrito}
\begin{mdframed}[nobreak]
  \section{\mytitle}
  \begin{tabular}{l}
  {\sffamily\footnotesize \textbf{keywords:}
    indian, main, chicken, fusion, mexican } \vspace{1ex}\\
  \begin{minipage}[t]{0.35\textwidth}
    {\sffamily\bfseries Ingredients}\vspace{0.5ex}
    \begin{description}
      \item[Butter Chicken]\hfill
        \begin{itemize}
          \item 1.5 pounds boneless, skinless chicken thighs, cut into bite sized pieces
          \item 1 tablespoon neutral oil
          \item 1 tablespoon each ginger paste \& garlic paste
          \item 1.5 tablespoons red chili powder
          \item \nicefrac{1}{2} teaspoon each kosher salt \& pepper
          \item \nicefrac{1}{4} cup butter
          \item 1 small, or \nicefrac{1}{2} large yellow onion, diced
          \item 1 tablespoon each ginger \& garlic paste
          \item 1 tablespoon garam masala
          \item 2 tablespoons oil
          \item 1 small onion, diced
          \item 1 tablespoon garam masala
          \item 1 teaspoon ground cumin
          \item 1 teaspoon smoked paprika
          \item 1 teaspoon red chili powder
          \item 1 teaspoon ground turmeric
          \item \nicefrac{1}{4} cup tomato paste
          \item \nicefrac{1}{2} cup water
          \item 1 cup heavy cream
          \item 1 teaspoon each salt \& pepper
          \item Large flour tortillas, 12-14 inches
          \item Garlic butter rice, see recipe below
        \end{itemize}
      \item[Optional]\hfill
        \begin{itemize}\raggedright\small\sffamily
          \item Papadum, thin super light \& crispy Indian crackers
        \end{itemize}
    \end{description}
  \end{minipage}
  \qquad
  \begin{minipage}[t]{0.55\textwidth}
    {\sffamily\bfseries Method}\vspace{0.5ex}
    \begin{enumerate}\raggedright\small\sffamily
      \item Put chicken in a bowl with the oil, garlic and ginger paste,
        red pepper powder and salt and pepper – mix well to combine and set
        aside 15 minutes
      \item Put a large pan over high heat and when hot, add half the
        chicken – cook until just beginning to brown, about 3 minutes then
        remove to a plate and repeat with the rest of the chicken – put
        that batch on the plate too
      \item Turn the pan down to medium high and add the butter, and
        onion – stir until just softened about 3 minutes
      \item Then put in the garlic \& ginger, stir till fragrant then add
        remaining spices – mixing them into the onions well
      \item Add tomato paste and water – mix well to combine
      \item Add the cream and when relatively smooth and beautifully
        incorporated, put on a lid, turn down to a low simmer and let
        finish cooking – 4-5 minutes
      \item Remove lid, season to taste with salt \& pepper, and you’re
        ready to eat as is, or make into a burrito
      \item Warm tortilla to make it more pliable, then add some garlic
        butter rice, the butter chicken, some papadum and roll up – done!
    \end{enumerate}
  \end{minipage} \vspace{3ex}\\
  \multicolumn{1}{c}{\small\ttfamily
  \url{https://www.thecookingguy.com/cookbook/2023/6/9/indian-butter-chicken-burrito}} \\
\end{tabular}
\end{mdframed}

\end{document}
