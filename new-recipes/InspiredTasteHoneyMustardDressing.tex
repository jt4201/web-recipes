% vim: ft=tex:
\documentclass[../cookbook]{subfiles}
\begin{document}

\renewcommand{\mytitle}{Honey Mustard Dressing (Inspired Taste)}
\renewcommand{\myurl}{https://www.inspiredtaste.net/38249/honey-mustard-dressing-recipe/}

\sbox{\ingredients}{%
  \begin{minipage}[t]{0.32\textwidth}
    {\sffamily\bfseries Ingredients}\vspace{0.5ex}
    \begin{itemize}
      \item \nicefrac{1}{4} cup (65 grams) Dijon mustard
      \item 3 to 4 tablespoons (65 to 85 grams) honey, to taste
      \item 3 to 4 tablespoons (45 to 60 ml) apple cider vinegar
      \item \nicefrac{1}{4} cup (60 ml) extra virgin olive oil
      \item 1 tablespoon fresh lemon juice, optional
      \item \nicefrac{1}{4} to \nicefrac{1}{2} teaspoon fine sea salt
      \item \nicefrac{1}{4} teaspoon fresh ground black pepper
    \end{itemize}
  \end{minipage}
}

\sbox{\method}{%
  \begin{minipage}[t]{0.55\textwidth}
    {\sffamily\bfseries Method}\vspace{0.5ex}
    \begin{enumerate}
      \item Combine the mustard, honey, apple cider vinegar, lemon juice, \nicefrac{1}{4} teaspoon salt, and the
        black pepper and whisk until well blended and creamy. Whisk in the oil, and then taste the
        dressing. Adjust with additional salt, vinegar, or honey. I enjoy my dressing to taste more
        tart, so I typically add an extra tablespoon of vinegar.
      \item Alternatively, add all ingredients to a medium mason jar, secure the lid, and shake
        until blended.
      \item Keep tightly covered in the refrigerator for up to three weeks.
    \end{enumerate}
  \end{minipage}
}

\myRecipe{Sauces\&Condiments!Mustard}{\mytitle{}}

% \addcontentsline{toc}{subsubsection}{\mytitle}
% \subsubsection*{\mytitle}
\begin{tabular}{l}
  \usebox{\ingredients}\quad\usebox{\method}\vspace{3ex}\\
  \multicolumn{1}{c}{\small\ttfamily \url{\myurl}} \\
\end{tabular}

\end{document}
