\documentclass[web-recipes.tex]{subfiles}
\begin{document}

\renewcommand{\mytitle}{Lasagna Pie}
\renewcommand{\myurl}{https://www.sipandfeast.com/lasagna-pie/}
\begin{mdframed}[nobreak]
  \section{\mytitle}
  {\sffamily\footnotesize \textbf{keywords:}
  italian\index{cuisine!italian},
pasta\index{pasta}
main\index{course!main}
beef\index{meat!beef}
pork\index{meat!pork}
}
  \vspace{0.5ex}

  \begin{minipage}[t]{0.35\textwidth}
    {\sffamily\bfseries Ingredients}\vspace{0.5ex}
    \begin{description}
      \item[meat sauce] \hfill
        \begin{itemize}
          \item \nicefrac{1}{4} cup olive oil
          \item 1 medium onion diced
          \item 1 pound ground chuck
          \item 1 pound ground pork
          \item 5 cloves garlic minced
          \item \nicefrac{1}{2} teaspoon crushed red pepper flakes
          \item \nicefrac{1}{2} cup dry white wine
          \item 2 28-ounce cans plum tomatoes hand crushed or blender pulsed
          \item salt and pepper to taste
          \item 5 basil leaves chopped
        \end{itemize}
      \item[lasagna pie] \hfill
        \begin{itemize}
          \item 1 pound mafaldine pasta or broken lasagna noodles
          \item 4 cups meat sauce from above
          \item 3 large eggs beaten
          \item 1 teaspoon coarse cracked black pepper
          \item \nicefrac{3}{4} cup Pecorino Romano cheese grated, divided
          \item 1 cup ricotta
          \item 3 cups mozzarella shredded, divided
          \item 3 tablespoons basil chopped
          \item 3 tablespoons flat leaf Italian parsley minced
          \item 3 tablespoons olive oil
        \end{itemize}
    \end{description}
  \end{minipage}
  \qquad
  \begin{minipage}[t]{0.55\textwidth}
    {\sffamily\bfseries Method}\vspace{0.5ex}
    \begin{description}
      \item[meat sauce]\hfill
        \begin{enumerate}
          \item Heat a large pot or deep pan to medium and add the
            olive oil and onion. Saute until soft (4-5 minutes)
            then add the beef and pork.
          \item Turn the heat to medium-high and brown the meat.
            Remove the excess fat with a spoon, and once the meat
            is cooked through add the garlic and cook until
            fragrant (another 2 minutes). Add the red pepper flakes
            and cook for 30 more seconds.
          \item Add the wine and cook for 2 minutes or until the
            alcohol smell dissipates. Add the tomatoes to the pot
            and mix well. Bring the sauce to a simmer, stirring
            frequently to avoid any sticking.
          \item Let the sauce simmer for at least 30 minutes. Taste
            test and adjust salt and pepper to taste and add the
            chopped basil.
        \end{enumerate}
      \item[lasagna pie]\hfill
        \begin{enumerate}
          \item Preheat oven to 375f and set a rack on the middle
            level and the other rack near the top.
          \item Bring a large pot of salted water to boil (2
            tablespoons salt per gallon) and cook the mafaldine
            until 2 minutes less than al dente.
          \item Meanwhile, in a large bowl, beat together the eggs,
            ricotta, Pecorino, black pepper. basil, and parsley.
            Add 4 cups of the meat sauce to the bowl and mix once
            more.
          \item Drain the pasta and add to the bowl along with 2
            cups of the shredded mozzarella and mix well.
          \item Heat a 12-inch oven-proof or ideally a cast iron
            pan to medium heat.
          \item Add 3 tablespoons of olive oil to the hot pan and
            spread to coat the whole pan. Add the pasta mixture to
            the pan and cook for 1 minute without stirring then
            turn off the heat.
          \item Top with the remaining mozzarella cheese and bake
            for 12 minutes in the center of the oven or until the
            lasagna pie is set. For a browner top, move the pan to
            the upper rack and broil for 2-3 minutes but watch
            carefully.
          \item Let the pie sit for 5 minutes before removing from
            the pan and cutting it into slices. Serve each slice
            with extra meat sauce and grated Pecorino. Enjoy!
        \end{enumerate}
    \end{description}
  \end{minipage}

  \vspace{3em}
  \centering{\small\ttfamily \myurl}
\end{mdframed}

\end{document}
