% vim: ft=tex:
\documentclass[../web-recipes.tex]{subfiles}
\begin{document}

\pagestyle{empty}

\renewcommand{\mytitle}{Quick-Pickled Onions (TheKitchn)}
\renewcommand{\myurl}{https://www.thekitchn.com/how-to-make-quickpickled-red-onions-cooking-lessons-from-the-kitchn-193247}

\myRecipe{Condiments}{\mytitle{}}

\sbox{\ingredients}{%
  \begin{minipage}[t]{\the\dimexpr 0.35\linewidth - 1em\relax}
    {\sffamily\bfseries Ingredients}\vspace{0.5ex}
    \begin{itemize}
      \item 1 medium red onion, about 5 ounces
      \item \nicefrac{1}{2} teaspoon sugar
      \item \nicefrac{1}{2} teaspoon salt
      \item \nicefrac{3}{4} cup rice vinegar, white vinegar, or apple cider vinegar
      \item \textbf{optional}
        \begin{itemize}
          \item 1 small clove of garlic, halved
          \item 5 black peppercorns
          \item 5 allspice berries
          \item 3 small sprigs of thyme
          \item 1 small dried chili
        \end{itemize}
    \end{itemize}
  \end{minipage}
}

\sbox{\method}{%
  \begin{minipage}[t]{\the\dimexpr 0.6\linewidth - 1em\relax}
    {\sffamily\bfseries Method}\vspace{0.5ex}
    \begin{enumerate}
      \item \textbf{Slice the onions} Start 2 or 3 cups of water on to boil in a kettle. Peel and
        thinly slice the onion into approximately \nicefrac{1}{4}-inch moons. Peel and cut the garlic clove in
        half.
      \item \textbf{Dissolve the sugar and salt} In the container you will be using to store the
        onions, add the sugar, salt, vinegar, and flavorings. Stir to dissolve.
      \item \textbf{Par-blanch the onions} Place the onions in the sieve and place the sieve in the
        sink. Slowly pour the boiling water over the onions and let them drain.
      \item \textbf{Add the onions to the jar} Add the onions to the jar and stir gently to evenly
        distribute the flavorings.
      \item \textbf{Store} The onions will be ready in about 30 minutes, but are better after a few
        hours. Store in the refrigerator. They will keep for several weeks, but are best in the
        first week.
    \end{enumerate}
  \end{minipage}
}


  \begin{tabular}{l}
    \usebox{\ingredients}\hspace{1em}\usebox{\method}\vspace{3ex}\\
    \multicolumn{1}{c}{\small\ttfamily \url{\myurl}} \\
  \end{tabular}

\end{document}
