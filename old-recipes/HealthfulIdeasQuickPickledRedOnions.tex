% vim: ft=tex:
\documentclass[../web-recipes.tex]{subfiles}
\begin{document}

\pagestyle{empty}

\renewcommand{\mytitle}{Quick Pickled Red Onions (Healthful Ideas)}
\renewcommand{\myurl}{https://thehealthfulideas.com/quick-pickled-red-onions/}

\sbox{\ingredients}{%
  \begin{minipage}[t]{\the\dimexpr 0.35\linewidth - 1em\relax}
    {\sffamily\bfseries Ingredients}\vspace{0.5ex}
    \begin{itemize}
      \item 2 medium red onions, peeled and thinly sliced
      \item \nicefrac{1}{2} cup water
      \item \nicefrac{1}{4} cup champagne vinegar
      \item \nicefrac{1}{4} cup sherry vinegar
      \item \nicefrac{1}{2} tsp sea salt or Himalayan pink salt
      \item \textbf{optional}
        \begin{itemize}
          \item 1 tsp maple syrup
          \item \nicefrac{1}{2} tsp crushed red pepper flakes
          \item \nicefrac{1}{2} tsp mustard seeds
        \end{itemize}
    \end{itemize}
  \end{minipage}
}

\sbox{\method}{%
  \begin{minipage}[t]{\the\dimexpr 0.6\linewidth - 1em\relax}
    {\sffamily\bfseries Method}\vspace{0.5ex}
    \begin{enumerate}
      \item Start by peeling and slicing your onions. I like to slice them very thin but it’s up to
        you, if you have thicker slices, they will have a nice crunch.
      \item Slice your onion in half and thinly slice them vertically not horizontally, this will
        give you shorter pieces. I like to cut out the middle part to avoid having big chunks of
        pickle but this is optional. See post for photos.
      \item Tightly pack the sliced onions into a 2 cup heat proof mason jar and set aside.
      \item Add the water, champagne vinegar, sherry vinegar, maple syrup, sea salt, crushed red
        pepper flakes, and mustard seeds into a small pot. Stir and bring to a boil. Once it’s
        boiling, take it off the heat and pour it over the onion to the top.
      \item Some of the onions might stick out, use a spoon to gently push them down to get them
        into the liquid, careful not to spill over, it’s very hot. The onions will shrink and wilt
        down a little as they sit so don’t stress it if some are sticking out too much.
      \item Let the onions sit in the liquid on the counter until they’ve cooled down or for at
        least 30 mins.
      \item Once completely cool, cover with a tight lid and store in the fridge for up to 2 weeks.
    \end{enumerate}
  \end{minipage}
}

\myRecipe{Condiments}{\mytitle{}}

% \addcontentsline{toc}{subsubsection}{\mytitle}
%   \subsubsection*{\mytitle}
  \begin{tabular}{l}
    \usebox{\ingredients}\hspace{1em}\usebox{\method}\vspace{3ex}\\
    \multicolumn{1}{c}{\small\ttfamily \url{\myurl}} \\
  \end{tabular}
\end{document}
