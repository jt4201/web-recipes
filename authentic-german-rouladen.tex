\documentclass[web-recipes.tex]{subfiles}
\begin{document}

\renewcommand{\myurl}{https://www.daringgourmet.com/rouladen-recipe/}
\renewcommand{\mytitle}{Authentic German Rouladen (Daring Gourmet)}
\myRecipe{Mains!Rouladen}{\mytitle{}}

% \textbf{\Large \mytitle{}}
% \mainlist{main}{\mytitle}
% \addcontentsline{toc}{subsubsection}{\mytitle}

      \begin{tabular}{l}
        \begin{minipage}[t]{0.35\textwidth}
          {\sffamily\bfseries Ingredients}\vspace{0.5ex}
          \begin{description}
            \item[For the Rouladen] \hfill
              \begin{itemize}
              \item 8 slices top round beef, about 4x6 inches in size and
                \nicefrac{1}{4} inch thick
              \item \nicefrac{1}{3} cup German yellow mustard
              \item 8 slices bacon
              \item 8 medium German pickles , sliced lengthwise
              \item 1 medium yellow onion , chopped
              \item salt and freshly ground black pepper
              \end{itemize}
              \item[For the Gravy] \hfill
                \begin{itemize}
                \item 1 tablespoon butter
                \item 1 tablespoon cooking oil
                \item 1 medium yellow onion , chopped
                \item 1 clove garlic , minced
                \item 1 small leek , chopped, rinsed and drained in colander
                \item 1 large carrot , chopped
                \item 1 large celery stalk , chopped
                \item 1 cup dry red wine
                \item 2 cups strong beef broth
                \item 1 tablespoon tomato paste
                \item 1 bay leaf
                \item 1 teaspoon sugar
                \item \nicefrac{1}{2} teaspoon salt
                \item \nicefrac{1}{4} teaspoon freshly ground black pepper
                \item 4 tablespoons chilled butter
                \item cornstarch or flour dissolved in a little water for thickening

                  (1-2 tablespoons cornstarch dissolved in 2-3 tablespoons water
                  or 2-3 tablespoons flour dissolved in \nicefrac{1}{4} - \nicefrac{1}{3} cups water)
              \end{itemize}
          \end{description}
        \end{minipage}
        \qquad
        \begin{minipage}[t]{0.55\textwidth}
          {\sffamily\bfseries Method}\vspace{0.5ex}
          \begin{description}
            \item[For the Rouladen] \hfill
            \begin{enumerate}
            \item Lay the beef slices out on a work surface. Spread each beef slices with about 2
              teaspoons of mustard and sprinkle with a little salt and freshly ground black pepper. Place
              a strip of bacon on each beef slice so it's running the same length as the beef. Place the
              sliced German pickles and chopped onions on each beef slice. Roll up the beef slices,
              tucking in the sides as best you can and securing the beef rolls with toothpicks or cooking
              twine.
            \item Heat the butter and oil in a heavy Dutch oven or pot (make sure
              it's oven-safe if baking in the oven) and generously brown the rouladen
              on all sides. Browning them well will ensure a rich and flavorful
              gravy. Set the rouladen aside on a plate.
            \item *Do not remove the browned bits in the bottom of the pan (important
              for a flavorful gravy): Add the onions to the pot and a little more
              butter or oil if needed. Cook the onions until softened and
              translucent, about 5 minutes. Add the garlic and cook for another
              minute. Add the leek, carrots and celery and cook for another 5
              minutes. Pour in the red wine, bring to a rapid boil for one minute,
              reduce the heat to medium and simmer for 2-3 more minutes. Add the beef
              broth, tomato paste, bay leaf, sugar, salt and pepper.
            \item Nestle the beef rouladen in the pot.

              Oven or Stovetop: You can cook the rouladen, covered, on the stovetop
              on low for about 90 minutes or until fork tender, but for the most even
              cooking we recommend transferring the pot (make sure it's oven-safe) to
              the oven preheated to 325 F and cook it there for about 90 minutes or
              until fork tender.
            \end{enumerate}
            \item[For the Gravy] \hfill
              \begin{enumerate}
              \item When the beef is fork tender, remove the rouladen from the pot and
                set aside. Pour the liquid and vegetables through a strainer and
                reserve the liquid. (You can eat the veggies on the side or puree them
                in the blender and then return them to the gravy.) Return the strained
                liquid back to the pot and bring to a simmer. Thicken the gravy either
                with either a cornstarch slurry (for a clear/translucent gravy) or
                flour slurry (for an opaque gravy). For a creamy gravy you can also add
                a few tablespoons of heavy cream at this point. Simmer, whisking
                constantly, until the gravy is thickened. Add the chilled butter,
                whisking constantly, until the butter is melted and incorporated. Add
                salt, pepper and mustard to taste. Note: If you prefer a creamy gravy
                you can stir in some heavy cream. Carefully remove the toothpicks or
                cooking twine from the rouladen and return them to the gravy and heat
                through.
              \item Serve the rouladen and gravy with Homemade Rotkohl and either
                Homemade Spätzle, Homemade SemmelKnödel (or Kartoffelknödel) or boiled
                potatoes.
              \end{enumerate}
          \end{description}
        \end{minipage} \vspace{3ex}\\
        \multicolumn{1}{c}{\small\ttfamily \url{\myurl}} \\
      \end{tabular}
    \end{document}
