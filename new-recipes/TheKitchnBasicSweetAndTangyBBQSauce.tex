% vim: ft=tex:
\documentclass[../cookbook]{subfiles}
\begin{document}

\pagestyle{empty}

\renewcommand{\mytitle}{Basic Sweet \& Tangy BBQ Sauce (The Kitchn)}
\renewcommand{\myurl}{https://www.thekitchn.com/recipe-basic-sweet-tangy-bbq-sauce-recipes-from-the-kitchn-83487}

\myRecipe{Sauces\&Condiments!BBQ}{\mytitle{}}

\sbox{\ingredients}{%
  \begin{minipage}[t]{\the\dimexpr 0.35\linewidth - 1em\relax}
    {\sffamily\bfseries Ingredients}\vspace{0.5ex}
    \begin{itemize}
      \item \nicefrac{1}{2} tablespoon olive oil
      \item \nicefrac{1}{4} medium red onion, diced
      \item 1 clove garlic, minced
      \item 1 tablespoon tomato paste
      \item \nicefrac{1}{2} teaspoon cumin
      \item 1 can (8-ounce) can tomato puree or sauce (no spices)
      \item 2 tablespoons packed brown sugar
      \item 1 tablespoon cider vinegar
      \item 1 tablespoon molasses
      \item 2 teaspoons Worcestershire sauce
      \item 1 teaspoon Dijon or brown mustard
      \item 1 teaspoon salt
      \item Freshly ground pepper
      \item \textbf{optional}
        \begin{itemize}
          \item 1 teaspoon liquid smoke
          \item A few dashes hot sauce
        \end{itemize}
    \end{itemize}
  \end{minipage}
}

\sbox{\method}{%
  \begin{minipage}[t]{\the\dimexpr 0.6\linewidth - 1em\relax}
    {\sffamily\bfseries Method}\vspace{0.5ex}
    \begin{enumerate}
      \item Heat a splash of olive oil in a medium sauce pan over medium heat. Add the onions and
        cook until soft, about 5 minutes. Add the garlic and cook for another minute or two.
      \item Reduce the heat to low and mix in the tomato paste and cumin. Add the tomato puree and
        all remaining ingredients. Stir until combined and simmer for 5 to 10 minutes, until
        thickened to your liking. Taste and adjust salt, pepper, or other seasonings as you see fit.
      \item Transfer the sauce to a blender or use an immersion blender to blend until smooth. Add
        more water, a tablespoon or two at a time, if you prefer a thinner sauce.
      \item This sauce will keep refrigerated for about 2 weeks or can be frozen for up to 3 months.
    \end{enumerate}
  \end{minipage}
}


  \begin{tabular}{l}
    \usebox{\ingredients}\hspace{1em}\usebox{\method}\vspace{3ex}\\
    \multicolumn{1}{c}{\small\ttfamily \url{\myurl}} \\
  \end{tabular}

\end{document}
