% nop
% arara: xelatex
% arara: xelatex
% arara: latexmk: { options: ['-c'] }
% vim: tw=86:
\documentclass[../cookbook]{subfiles}
\graphicspath{{\subfix{../images/}}}
\begin{document}

\renewcommand{\myTitleText}{Cincinnati Chili}
\renewcommand{\myURLText}{https://www.thekitchn.com/cincinnati-chili-recipe-23565738}
\renewcommand{\myFilename}{tK-cincinnati-chile.jpg}
\renewcommand{\myServings}{6}
\renewcommand{\myAuthor}{thekitchn}
\renewcommand{\myCookTime}{1 hour 10 minutes}
\renewcommand{\myPrepTime}{10 minutes}
\renewcommand{\myTotalTime}{1 hour 20 minutes}
\renewcommand{\myIndexEntry}{Mains}
\renewcommand{\myPreamble}{
    This Midwestern specialty has beans and is typically served over a bed of spaghetti.
}

\hypersetup{%
  pdfauthor={\myAuthor},
  pdftitle={\myTitleText}
}

\sbox{\myIngred}{%
    \begin{minipage}[t]{\the\dimexpr0.35\textwidth}\raggedright\sffamily\small
        \textbf{\large Ingredients}
        \begin{jitem}
            \item For the chili
                \begin{jsubitem}
                \item 1 medium yellow onion
                \item 4 cloves garlic
                \item 1 tablespoon olive oil
                \item 1 pound lean ground beef
                \item 1 1/4 teaspoons kosher salt, plus more as needed
                \item 3 tablespoons natural unsweetened cocoa powder
                \item 2 tablespoons chili powder
                \item 2 teaspoons ground cinnamon
                \item 1 teaspoon ground allspice
                \item 1 teaspoon ground cumin
                \item {\bfseries optional}
                    \begin{jsubitem}
                \item 1/8 teaspoon cayenne pepper
                \end{jsubitem}
                \item 1 (15-ounce) can tomato sauce
                \item 2 cups water
                \item 1 tablespoon Worcestershire sauce
                \item 1 tablespoon apple cider vinegar
                \item 2 bay leaves
                \end{jsubitem}
            \item Serving options
                \begin{jsubitem}
                \item 1 pound dried spaghetti
                \item 1 (about 15-ounce) can kidney beans
                \item Finely grated sharp cheddar cheese, preferably yellow
                \item Small-diced yellow onion
                \end{jsubitem}
            \end{jitem}
        \end{minipage}
    }

    \sbox{\myMethod}{%
        \begin{minipage}[t]{\the\dimexpr0.6\textwidth-2em}\raggedright\sffamily\small
            \textbf{\large Method}
            \begin{jnum}
            \item Finely chop 1 medium yellow onion (about 2 cups) and mince 4 garlic
                cloves (about 2 teaspoons).
            \item Heat 1 tablespoon olive oil in a large high-sided skillet or Dutch
                oven over medium-high heat until shimmering. Add the onion, garlic, 1
                pound lean ground beef, and 1 1/4 teaspoons kosher salt. Cook,
                breaking up the beef into small pieces with a wooden spoon and
                stirring occasionally, until the beef is no longer pink, about 8
                minutes.
            \item Meanwhile, place 3 tablespoons natural cocoa powder, 2 tablespoons
                chili powder, 2 teaspoons ground cinnamon, 1 teaspoon ground allspice,
                1 teaspoon ground cumin, and 1/8 teaspoon cayenne pepper if desired in
                a small bowl.
            \item Sprinkle the spice mixture over the beef and cook, stirring
                constantly, until fragrant, about 1 minute. Add 1 (15-ounce) can
                tomato sauce, 2 cups water, 1 tablespoon Worcestershire sauce, 1
                tablespoon apple cider vinegar, and 2 bay leaves. Stir to combine and
                bring to a boil.
            \item Reduce the heat to maintain a simmer. Cook, stirring every 10
                minutes or so, until the flavors meld and the chili thickens to a meat
                sauce consistency, about 45 minutes.
            \item About 15 minutes before the chili is ready, bring a large pot of
                heavily salted water to a boil. Add 1 pound dried spaghetti to the
                boiling water and cook according to package directions until just
                tender. Drain well, drizzle with a little olive oil, and toss to coat.
            \item Taste the chili and season with more kosher salt as needed. It
                should be heavily seasoned since it will be served on spaghetti.
                Remove and discard the bay leaves. Drain 1 (about 15-ounce) can kidney
                beans and rinse under hot tap water; drain again and transfer to a
                serving bowl.
            \item Serve the chili over cooked spaghetti topped with finely grated
                cheddar cheese, small-diced yellow onion, and kidney beans if desired.
            \end{jnum}
        \end{minipage}
    }

\newcommand{\cmdSubText}[2]{%
  \ifthenelse{\equal{#2}{}} {} {\cmdSubtitle{#1}{#2}}
}

\sbox{\myTitle}{%
  \begin{minipage}[t]{0.6\textwidth}\raggedright%
    \textbf{\LARGE\bfseries \myTitleText} \vspace{2ex} \\
    \cmdSubText{PREP~TIME}{\myPrepTime}\quad
    \cmdSubText{COOK~TIME}{\myCookTime}\quad
    \cmdSubText{TOTAL~TIME}{\myTotalTime}\quad \\
    \cmdSubText{SERVINGS}{\myServings}\vspace{0.5ex} \\
    \cmdSubText{AUTHOR}{\myAuthor}\\
    \cmdSubText{SOURCE}{\href{\myURLText}{\myTitleText} \vspace{1ex} \\}
    \myPreamble{}
  \end{minipage}
}

\sbox{\myImage}{%
  \begin{minipage}[t]{0.35\textwidth}
    \ifthenelse{\equal{\myFilename}{}}{}{\adjustbox{valign=t}{\includegraphics[width=\textwidth]{\myFilename}}}
  \end{minipage}
}

\sbox{\myNotes}{%
    \begin{minipage}[t]{0.6\textwidth}
        \textbf{\large Notes}
        \begin{description}
            \item [Two-way] Spaghetti topped with chili (the basics)
            \item [Three-way] Spaghetti, chili, and finely grated cheddar cheese (lots of it!)
            \item [Four-way] Spaghetti, chili, minced onion, and cheese on top
            \item [Five-way] Spaghetti, chili, minced onions, kidney or chili beans, and cheese on top
    \end{description}
    \end{minipage} 
}

\framebox{%
  \cmdRecipe{recipelist}{\myIndexEntry}{\myTitleText}
  \cmdRecipe{authorlist}{\myAuthor}{\myTitleText}
\begin{tabular}{@{}llllll@{}}
  \multicolumn{4}{@{}l@{}}{\usebox{\myTitle}} &
  \multicolumn{2}{@{}l@{}}{\usebox{\myImage}} \\
  \multicolumn{6}{@{}l@{}}{} \rule{0.95\linewidth}{0.4pt}\vspace{2ex} \\ 
  \multicolumn{2}{@{}l}{\usebox{\myIngred}} &
  \multicolumn{4}{@{}l}{\usebox{\myMethod}} \vspace{2ex}\\
  \multicolumn{6}{@{}l}{\usebox{\myNotes}} \\
\end{tabular}
}

\end{document}
