% arara: xelatex
\documentclass[../cookbook]{subfiles}
\begin{document}

\renewcommand{\myAuthor}{ChefSteps}
\renewcommand{\myTitletext}{Nacho Cheese}
\renewcommand{\myUrltext}{https://www.chefsteps.com/activities/nacho-cheese}
\renewcommand{\myServingstext}{12}
\renewcommand{\mycourse}{Sauces\&Condiments!Cheese}

\hypersetup{
  pdftitle = {\myTitletext},
  pdfauthor = {\myAuthor}
}

\sbox{\Ingredients}{%
  \begin{minipage}[t]{\the\dimexpr 0.4\textwidth - 2em}
    \par\mySubtitle{Ingredients}
    \par\begin{itemize}
      \item 300 g Tillamook extra sharp cheddar
      \item 300 g Milk
      \item 12 g Sodium citrate
      \item 4 g Salt
      \item 1.5 g Sodium hexametaphosphate , SHMP
      \item 50 g Pickled jalapeños 
    \end{itemize}
  \end{minipage}
}

\sbox{\Instructions}{%
  \begin{minipage}[t]{\the\dimexpr 0.6\textwidth - 2em}
    \par\mySubtitle{Instructions}
    \par\begin{enumerate}
      \item Preheat Joule to 167 °F / 75 °C
      \item Cut cheese into small pieces
      \item Place milk and salts in a ziplock-style bag, and agitate slightly to dissolve.
        Add cheese, and seal bag. Drape the bag over the side of the pot, taking care not
        to let any water leak into the bag.
      \item Place bag in the heated water for 15 minutes to melt the cheese.
      \item Pour nacho cheese into an insulated container (like a cocotte or other pot
        with a lid), and serve warm.lace melted cheese and pickled jalapeños in blender,
        and blend until smooth. If you want larger chunks of jalapeños, chop them by hand
        and fold them in after you blend.
    \end{enumerate}
  \end{minipage}
}

\framebox{
  \cmdRecipe{recipelist}{\mycourse}{\myTitletext}
  \begin{tabular}{ll}
    \multicolumn{2}{l}{\LARGE\bfseries \myTitletext (\myAuthor)} \\ \addlinespace[1mm]
    \multicolumn{2}{l}{Servings: \myServings{\myServingstext}} \\
  \multicolumn{2}{l}{\myUrl{\myUrltext}{\myTitletext}} \\ \addlinespace[3mm]
  \usebox{\Ingredients} & \usebox{\Instructions} \\ \addlinespace[3mm]
\end{tabular}
}

\end{document}
