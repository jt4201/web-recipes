% vim: ft=tex:
\documentclass[web-recipes.tex]{subfiles}
\begin{document}

\pagestyle{empty}

\renewcommand{\mytitle}{Mini Apple Galettes (This Home Kitchen)}
\renewcommand{\myurl}{https://thishomekitchen.com/mini-apple-galettes/}

\myRecipe{Desserts}{\mytitle{}}

\sbox{\ingredients}{%
  \begin{minipage}[t]{\the\dimexpr 0.35\linewidth - 1em\relax}
    {\sffamily\bfseries Ingredients}\vspace{0.5ex}
    \begin{itemize}
      \item \textbf{Buttermilk Pie Crust}
        \begin{itemize}
          \item 6 ounces all-purpose flour (1 \nicefrac{1}{4} cup)
          \item 1 teaspoon sugar
          \item \nicefrac{1}{2} teaspoon salt
          \item 4 ounces unsalted butter cold
          \item \nicefrac{1}{4} cup buttermilk cold
      \end{itemize}
      \item \textbf{Apple Filling}
        \begin{itemize}
          \item 2 medium apples
          \item \nicefrac{1}{3} cup granulated sugar
          \item \nicefrac{1}{4} cup all-purpose flour
          \item 2 teaspoons cinnamon
          \item 1 teaspoon vanilla extract
      \end{itemize}
      \item \textbf{Assembling}
        \begin{itemize}
          \item 2 tablespoons melted butter
          \item granulated sugar for sprinkling
      \end{itemize}
      \end{itemize}
    \end{minipage}
  }

  \sbox{\method}{%
    \begin{minipage}[t]{\the\dimexpr 0.6\linewidth - 1em\relax}
      {\sffamily\bfseries Method}\vspace{0.5ex}
      \begin{enumerate}
        \item \textbf{Buttermilk Pie Crust}
          \begin{enumerate}
            \item Cut the butter into \nicefrac{1}{2} inch cubes. Keep the cubed butter and
              buttermilk in the fridge until ready to use.
            \item In a bowl, whisk together the flour, sugar, and salt.
            \item Add the cubed butter to the flour mixture and toss to coat. Using your thumbs
              against your index fingers, press the butter cubes into planks. Toss them with the
              flour as you work. Stop once you've pressed all of the butter and have a mixture of
              pea size to walnut-half size butter pieces.
            \item Add the buttermilk and mix. If the dough clumps easily, you have enough liquid. If
              not, add a teaspoon more buttermilk until the dough does clump easily.
            \item Dump the dough out onto a lightly floured surface and knead gently by folding it
              over itself 2 to 3 times.
            \item Shape the dough into a disk and wrap it in a towel. Chill the dough while you
              prepare the apples.
          \end{enumerate}
        \item \textbf{Apple Filling}
          \begin{enumerate}
            \item In a bowl, whisk together the sugar, flour, and cinnamon.
            \item Peel, core, and dice the apples into \nicefrac{1}{2} inch cubes.
            \item Add the diced apples and vanilla extract to the cinnamon sugar mixture and toss to
              coat. Set aside.
          \end{enumerate}
        \item \textbf{Assembling the Galettes}
          \begin{enumerate}
            \item Line a baking sheet with parchment paper.
            \item Remove the dough from the fridge and place it on a lightly floured work surface.
            \item Using a lightly floured rolling pin, start in the center and roll out. Rotate and
              roll from the center out. Continue this until the dough is about \nicefrac{1}{8}-inch thick and
              about 12 inches across.
            \item Cut 4-inch rounds out of the dough. Press the scraps together and roll back out to
              get another 4 rounds.
            \item Spoon 1 tablespoon of apples onto the center of one of the dough rounds leaving
              space around the edges.
            \item Fold up the edges and pleat as you go, pressing down to secure them. Transfer the
              mini galette to the parchment-lined baking sheet and continue with the rest of the
              galettes.
            \item Freeze the galettes for 1 hour.
          \end{enumerate}
        \item \textbf{Baking the Galettes}
          \begin{enumerate}
            \item Preheat the oven to 375 degrees F.
            \item Transfer the frozen galettes and parchment paper to a new baking sheet.
            \item Brush one galette with the melted butter then immediately sprinkle with sugar.
              Continue with the rest of the galettes.
            \item Bake for 40 to 50 minutes, or until the crust is golden brown and the apples are
              softened.
            \item Serve warm with vanilla ice cream.
          \end{enumerate}
      \end{enumerate}
    \end{minipage}
  }


  \begin{tabular}{l}
    \usebox{\ingredients}\hspace{1em}\usebox{\method}\vspace{3ex}\\
    \multicolumn{1}{c}{\small\ttfamily \url{\myurl}} \\
  \end{tabular}

\end{document}
