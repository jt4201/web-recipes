% arara: xelatex
% arara: xelatex
% vim: tw=86:
\documentclass[oneside]{article}
% \usepackage{showframe}
\usepackage[noheadfoot,nomarginpar,margin=5mm]{geometry}
\pagestyle{empty}

\setlength{\parindent}{0em}

\newsavebox{\myIngred}
\newsavebox{\myMethod}
\newsavebox{\myNotes}

\newcommand{\myTitleText}{}
\newcommand{\myURLText}{}
\newcommand{\myServings}{}
\newcommand{\myPrepTime}{}
\newcommand{\myCookTime}{}
\newcommand{\myTotalTime}{}
\newcommand{\myAuthor}{}
\newcommand{\myPreamble}{}
\newcommand{\myFilename}{}

\usepackage[usenames,dvipsnames]{color}

\newcommand{\cmdSubtitle}[2]{%
{\sffamily {\color{Gray} #1:} #2}}

\usepackage{fontspec}
\setmainfont{TeX Gyre Bonum}

\usepackage{array}
\usepackage{tabularx}

\usepackage{graphicx}
\graphicspath{ {./images/} }

\usepackage{multirow}
\usepackage{adjustbox}

\usepackage{hyperref}
\hypersetup{
  colorlinks,
  urlcolor=blue,
  citecolor=black,
  linkcolor=ForestGreen
}

\begin{document}

\renewcommand{\myTitleText}{Apple Sausage Breakfast Sandwich}
\renewcommand{\myURLText}{https://www.freshoffthegrid.com/apple-maple-sausage/}
\renewcommand{\myFilename}{fotg-apple-sausage-breakfast-sandwich}
\renewcommand{\myServings}{8~sandwiches}
\renewcommand{\myAuthor}{Fresh~Off~The~Grid}
\renewcommand{\myCookTime}{10~minutes}
\renewcommand{\myPrepTime}{10~minutes}
\renewcommand{\myTotalTime}{20~minutes}
\renewcommand{\myPreamble}{%
  These Apple Maple Sausage Breakfast Sandwiches are a great make ahead camping meal.
  Sweet and savory, these sausage patties are full of great flavor and SO easy to make
  at the campsite! 
}

\sbox{\myIngred}{
  \begin{minipage}[t]{\the\dimexpr 0.35\textwidth}\raggedright
    \textbf{Ingredients}
    \begin{itemize}
      \item 1 apple
      \item 1 lemon
      \item 1 pound ground pork
      \item 1 tablespoon maple syrup
      \item 1 teaspoon dried sage
      \item 1 teaspoon fennel seeds
      \item 1 teaspoon dried thyme
      \item 1 teaspoon sea salt salt
      \item ¼ teaspoon smoked paprika
      \item ¼ teaspoon cinnamon
      \item \textbf{For the sandwiches}
        \begin{itemize}
          \item 8 Swiss cheese slices
          \item 8 English muffins
          \item 8 eggs
          \item Butter to serve
        \end{itemize}
    \end{itemize}
  \end{minipage}
}

\sbox{\myMethod}{
  \begin{minipage}[t]{\the\dimexpr 0.55\textwidth - 2em}\raggedright
    \textbf{Method}
    \begin{enumerate}
      \item \textbf{At Home}
        \begin{enumerate}
          \item Coarsely grate the apple on a box grater into a medium bowl. Add the
            juice of one lemon and enough cold water to cover. Set aside for 5
            minutes, then drain and squeeze out the excess water.
          \item Return the apple to the bowl and add the ground pork, maple syrup,
            spices, herbs, and salt. Mix well until everything is thoroughly combined.
            Divide the mix into 8 portions and form into thin discs about ¼ inch thick
            and 4 inches wide. If making these ahead of time, store in an air-tight
            container between squares of parchment for up to two days.
        \end{enumerate}
      \item \textbf{At Camp}
        \begin{enumerate}
          \item Heat up a griddle and cook the sausages 2-3 minutes on each side,
            until browned and cooked through. While the sausages are cooking, cook the
            eggs to your preference and toast the English muffins. To assemble, butter
            the muffins, then add the sausage, an egg, and a slice of swiss. Enjoy!
        \end{enumerate}
    \end{enumerate}
  \end{minipage}
}

\begin{tabular}{ll}
  \begin{minipage}[t]{0.6\textwidth}\raggedright
    \textbf{\LARGE\bfseries \myTitleText} \vspace{3mm} \\
    \cmdSubtitle{PREP~TIME}{\myPrepTime}\quad
    \cmdSubtitle{COOK~TIME}{\myCookTime} \\
    \cmdSubtitle{TOTAL~TIME}{\myTotalTime}\quad
    \cmdSubtitle{SERVINGS}{\myServings} \\
    \cmdSubtitle{AUTHOR}{\myAuthor} \\
    \cmdSubtitle{SOURCE}{\href{\myURLText}{\myTitleText}} \vspace{1ex} \\
    \myPreamble
\end{minipage} &
\begin{minipage}[t]{0.35\textwidth}
  \adjustbox{valign=t}{\includegraphics[width=\textwidth]{\myFilename}}
\end{minipage} \\
\end{tabular}\vspace{5mm}

\par\noindent\rule{0.95\textwidth}{0.4pt}\vspace{5mm}

\begin{tabular}{ll}
  \usebox{\myIngred} & \usebox{\myMethod} \\
\end{tabular}

\end{document}
