% arara: testing
% arara: xelatex
% arara: xelatex
% arara: latexmk: { options: ['-c'] }
% vim: tw=86:
\documentclass{newRecipe}
\begin{document}

\renewcommand{\myTitleText}{One Pot Beef Stroganoff}
\renewcommand{\myURLText}{https://www.freshoffthegrid.com/one-pot-beef-stroganoff/}
\renewcommand{\myFilename}{fotg-one-pot-beef-stroganoff.jpg}
\renewcommand{\myServings}{4 servings}
\renewcommand{\myAuthor}{Fresh Off The Grid}
\renewcommand{\myCookTime}{25 minutes}
\renewcommand{\myPrepTime}{5 minutes}
\renewcommand{\myTotalTime}{30 minutes}
\renewcommand{\myPreamble}{
  This beef stroganoff is a great one-skillet meal that will really elevate your camp
  dining experience!
}

\sbox{\myIngred}{
  \begin{minipage}[t]{\the\dimexpr 0.35\textwidth}\raggedright
    \textbf{Ingredients}
    \begin{itemize}
      \item ½ pound strip steak sliced into bite sized pieces
      \item 1 teaspoon salt
      \item 2 tablespoons oil
      \item 8 oz mushrooms quartered
      \item 1 small brown onion diced
      \item 2 cloves garlic minced
      \item 1 ½ cups broth
      \item 1 tablespoon worcestershire sauce
      \item 1 teaspoon thyme
      \item ¼ lb wide egg noodles
      \item ½ cup full fat sour cream, divided
    \end{itemize}
  \end{minipage}
}

\sbox{\myMethod}{
  \begin{minipage}[t]{\the\dimexpr 0.55\textwidth - 2em}\raggedright
    \textbf{Method}
    \begin{enumerate}
      \item Quarter the mushrooms, dice the onion, and mince the garlic.
      \item Heat the oil in a heavy-bottomed 10” skillet until smoking hot. Season the
        strip steak with salt \& pepper. Add the steak to the skillet and cook, until
        browned on both sides and cooked through. Remove from the skillet, tent with
        foil, and allow it to rest while preparing the vegetables and noodles.
      \item Reduce heat to medium. Add the mushrooms and saute 5 minutes, stirring
        infrequently. Add the onions and continue to saute an additional 5 minutes.
        Add garlic and saute 1 minute.
      \item Add the broth, Worcestershire sauce, and thyme. Use a wooden spoon or a
        spatula to scrape up any brown bits on the bottom of the skillet. Bring the
        liquid to a boil over high heat. Add the noodles and cook according to package
        instructions (this will depend on the noodles you buy - ours took 8 minutes),
        stirring occasionally to ensure the noodles cook evenly.
      \item Once the noodles are tender, remove skillet from heat. Temper half the
        sour cream with some of the sauce by adding a few spoonfuls of broth to the
        sour cream and stirring - this will raise the temperature of the sour cream
        and help prevent it from curdling when added to the pan. Mix the tempered sour
        cream to the pan, stir, then add the rest of the sour cream and stir to
        combine.
      \item Slice the cooked steak into bite-sized pieces, add to skillet, season to
        taste and serve!
    \end{enumerate}
  \end{minipage}
}

\begin{tabular}{ll}
  \begin{minipage}[t]{0.6\textwidth}\raggedright
    \ifthenelse{\equal{\myTitleText}{}} {}%
    {\textbf{\LARGE\bfseries \myTitleText} \vspace{2ex} \\ }
    \ifthenelse{ \equal{\myPrepTime}{} } {}%
      {\cmdSubtitle{PREP~TIME}{\myPrepTime}\quad}
      \ifthenelse{ \equal{\myCookTime}{} } {}%
      {\cmdSubtitle{COOK~TIME}{\myCookTime} \\ }
      \ifthenelse{ \equal{\myTotalTime}{} } {}%
    {\cmdSubtitle{TOTAL~TIME}{\myTotalTime}\quad}
      \ifthenelse{ \equal{\myServings}{} } {}%
      {\cmdSubtitle{SERVINGS}{\myServings} \\ }
      \ifthenelse{ \equal{\myAuthor}{} } {}%
      {\cmdSubtitle{AUTHOR}{\myAuthor} \\ }
      \ifthenelse{ \equal{\myURLText}{} } {}%
      {\cmdSubtitle{SOURCE}{\href{\myURLText}{\myTitleText} \vspace{1ex}} \\ }
    \ifthenelse{ \equal{\myPreamble}{}} {}%
      {\myPreamble}
\end{minipage} &
\begin{minipage}[t]{0.35\textwidth}
  \adjustbox{valign=t}{\includegraphics[width=\textwidth]{\myFilename}}
\end{minipage} \\
\end{tabular}\vspace{5mm}

\par\noindent\rule{0.95\textwidth}{0.4pt}\vspace{5mm}

\begin{tabular}{ll}
  \usebox{\myIngred} & \usebox{\myMethod} \\
\end{tabular}

\end{document}
