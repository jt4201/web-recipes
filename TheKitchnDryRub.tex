% vim: ft=tex:
\documentclass[web-recipes.tex]{subfiles}
\begin{document}

\renewcommand{\mytitle}{Dry Rub (theKitchn)}
\renewcommand{\myurl}{https://www.thekitchn.com/how-to-make-dry-rub-139632}

\sbox{\ingredients}{%
  \begin{minipage}[t]{0.32\textwidth}
    {\sffamily\bfseries Ingredients}\vspace{0.5ex}
    \begin{itemize}
      \item 2 cups packed brown sugar
      \item 1/2 cup paprika
      \item 1/4 cup plus 2 tablespoons granulated garlic
      \item 1/4 cup granulated onion
      \item 1/4 cup kosher salt
      \item 1/4 cup freshly ground black pepper
      \item 1/4 cup ground cumin
      \item 1/4 cup dry mustard powder
      \item 2 tablespoons ground ancho or chipotle chile powder
      \item 2 tablespoons cayenne pepper
    \end{itemize}
  \end{minipage}
}

\sbox{\method}{%
  \begin{minipage}[t]{0.55\textwidth}
    {\sffamily\bfseries Method}\vspace{0.5ex}
    \begin{description}
      \item [Gather Ingredients]
      Since there's no oven or complicated procedure involved in the making of this rub, the
      assembly of everything is key. Forgetting one ingredient can make things taste a little off,
      so make sure everyone (spice wise rather) is in attendance!
    \item [Measure]
      Measure ingredients into prep bowls. We usually mix them into smaller bowls before adding them
      to the large bowl, just in case we measure incorrectly it's easier to rectify.
    \item [Mix]
      Add all spices to your large bowl and add brown sugar. Combine with a whisk, or toss all
      ingredients into a large zip top bag and shake, shake, shake!
    \item [To Use]
      Rub your mixture into the piece of meat which you'll be using. We suggest this rub on anything
      that once had legs. Don't be afraid of it getting messy, that's a given, just get it in all
      nooks and crannies incuding bony or fatty parts.
    \item [Wrap In Plastic]
      Wrap your meat in plastic for at least an hour, up to a day and let things mingle. Although
      you can grill, broil or bake things immediately, the longer the rub sits on the meat the more
      flavor will develop further into it. This can keep up to 72 hours if needs be, but is best
      right around the 24 hour mark! Enjoy!
    \end{description}
  \end{minipage}
}

\myRecipe{Spice Blends}{\mytitle{}}

% \addcontentsline{toc}{subsubsection}{\mytitle}
%   \subsubsection*{\mytitle}
  \begin{tabular}{l}
    \usebox{\ingredients}\quad\usebox{\method}\vspace{3ex}\\
    \multicolumn{1}{c}{\small\ttfamily \url{\myurl}} \\
  \end{tabular}
\end{document}
