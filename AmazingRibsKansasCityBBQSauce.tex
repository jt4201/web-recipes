% vim: ft=tex:
\documentclass[web-recipes.tex]{subfiles}
\begin{document}

\pagestyle{empty}

\renewcommand{\mytitle}{Kansas City Style BBQ Sauce (Amazing Ribs)}
\renewcommand{\myurl}{https://amazingribs.com/tested-recipes/barbecue-sauce-recipes/kc-classic-bbq-sauce-recipe/}

\myRecipe{Sauces!BBQ}{\mytitle{}}

\sbox{\ingredients}{%
  \begin{minipage}[t]{\the\dimexpr 0.35\linewidth - 1em\relax}
    {\sffamily\bfseries Ingredients}\vspace{0.5ex}
    \begin{itemize}
      \item 2 tablespoons American chili powder
      \item 1 teaspoon ground black pepper
      \item 2 teaspoons Morton Kosher Salt
      \item 2 cups ketchup
      \item 1/2 cup yellow ballpark-style mustard
      \item 1/2 cup cider vinegar
      \item 1/3 cup Worcestershire sauce
      \item 1/4 cup lemon juice
      \item 1/4 cup steak sauce
      \item 1/4 cup dark molasses
      \item 1/4 to 3/4 cup honey (see note below)
      \item 1 teaspoon hot sauce
      \item 1 cup dark brown sugar (you can use light brown sugar if that’s all you have)
      \item 3 tablespoons vegetable oil
      \item 1 medium onion
      \item 4 medium cloves garlic
      \item \textbf{optional}
        \begin{itemize}
          \item 2 tablespoons tamarind paste
        \end{itemize}
    \end{itemize}
  \end{minipage}
}

\sbox{\method}{%
  \begin{minipage}[t]{\the\dimexpr 0.6\linewidth - 1em\relax}
    {\sffamily\bfseries Method}\vspace{0.5ex}
    \begin{enumerate}
      \item {Prep}. In a small bowl, mix the American chili powder, black pepper, and salt. In a large
        bowl, mix the ketchup, mustard, vinegar, Worcestershire, lemon juice, steak sauce, molasses,
        honey, hot sauce, and brown sugar. Mix them, but you don't have to mix thoroughly.
      \item Finely chop the onion and crush or mince the garlic cloves.
      \item {Cook}. Over medium heat, warm the oil in a large saucepan. Add the onions and sauté until
        limp and translucent, about 5 minutes. Crush the garlic, add it, and cook for another
        minute. Add the dry spices and stir for about 2 minutes to extract their oil-soluble
        flavors. Add the wet ingredients. Simmer over medium heat for 15 minutes with the lid off to
        thicken it a bit.
      \item {Taste and adjust}. Add more of anything that you want a little bit at a time. It may
        taste a bit vinegary at first, but that will be less obvious when you use it on meat. I
        recommend you run with my recipe the first time and then you can make it your own. Strain it
        if you don't want the chunks of onion and garlic. I like leaving them in, they give the
        sauce a home-made texture.
      \item {Serve}. You can use this sauce immediately as you would any other BBQ sauce, but I think
        it's better when aged overnight. You can store it into clean bottles in the refrigerator for
        a month or two.
    \end{enumerate}
  \end{minipage}
}


  \begin{tabular}{l}
    \usebox{\ingredients}\hspace{1em}\usebox{\method}\vspace{3ex}\\
    \multicolumn{1}{c}{\small\ttfamily \url{\myurl}} \\
  \end{tabular}

\end{document}
