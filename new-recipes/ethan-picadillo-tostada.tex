% nop
% arara: xelatex
% arara: xelatex
% arara: latexmk: { options: ['-c'] }
% vim: tw=86:
\documentclass[../cookbook]{subfiles}
\graphicspath{{\subfix{../images/}}}
\begin{document}

\renewcommand{\myTitleText}{Picadillo (Beef \& Potato) Tostada}
\renewcommand{\myURLText}{https://www.ethanchlebowski.com/cooking-techniques-recipes/picadillo-beef-amp-potato-tostada}
\renewcommand{\myFilename}{ethan-picadillo-tostada.jpg}
\renewcommand{\myServings}{1 tostada}
\renewcommand{\myAuthor}{Ethan Chlebowski}
\renewcommand{\myCookTime}{}
\renewcommand{\myPrepTime}{}
\renewcommand{\myTotalTime}{}
\renewcommand{\myIndexEntry}{Mains}
\renewcommand{\myPreamble}{
}

\hypersetup{%
  pdfauthor={\myAuthor},
  pdftitle={\myTitleText}
}

\sbox{\myIngred}{%
  \begin{minipage}[t]{\the\dimexpr0.35\textwidth}\sffamily\raggedright\small
    \textbf{\large Ingredients}
    \begin{itemize}[label=,leftmargin=1em,before=\bfseries]
      \item Tostada Components
        \begin{itemize}[label=\textbullet,leftmargin=1.5em,before=\mdseries]
        \item Tostada
        \item Picadillo
        \item Drizzle of crema
        \item Thinly shredded lettuce
        \item Diced tomatoes
        \item Pickled onions
        \item Crumbled queso fresco
        \end{itemize}
      \item Picadillo (makes enough for 4-6 tostadas)
        \begin{itemize}[label=\textbullet,leftmargin=1.5em,before=\mdseries]
        \item 1 lb (450 g) ground beef
        \item 1/4 white onion, diced
        \item 2 cloves of garlic, minced
        \item \textbf{Spices}
          \begin{itemize}[label=\textbullet,leftmargin=1.5em,before=\mdseries]
          \item 2 g (1 tsp) smoked paprika
          \item 3 g (1 tsp) ancho chili pepper
          \item Sprinkle of cinnamon
          \item ~3/4 lb (350 g) Yukon gold potatoes, diced (about equal parts volume with the cooked beef)
          \end{itemize}
        \item 1/2 can of whole peeled tomatoes (about 14 oz or 1.5 cups worth)
        \item Hot sauce of choice, to taste
        \end{itemize}
      \end{itemize}
    \end{minipage}
  }

  \sbox{\myMethod}{%
    \begin{minipage}[t]{\the\dimexpr0.6\textwidth-2em}\sffamily\raggedright\small
      \textbf{\large Method}
      \begin{itemize}[label=,leftmargin=1em,before=\bfseries]
        \item Start the potatoes
          \begin{itemize}[label=\textbullet,leftmargin=1.5em,before=\mdseries]
            \item To begin, chop the potatoes. You can wash them, but no need to peel them.
            \item Dice the potatoes into fairly small pieces by cutting them into planks, then strips, and finally cutting those across into cubes.
            \item Toss the diced potatoes into a pot, set them on a stove over high heat, and cover them with water. Lastly, add a generous sprinkle of salt.
          \end{itemize}
        \item Start the picadillo
          \begin{itemize}[label=\textbullet,leftmargin=1.5em,before=\mdseries]
            \item To prepare the beef, heat a stainless steel pan with a little oil over high heat. Add ground beef to the pan, press it down, and let it sear undisturbed until it starts browning.
            \item As it cooks, prep or gather any other ingredients.
            \item After 5 minutes, flip the beef and cook for 2 more minutes, breaking it up some. Transfer the cooked beef to a bowl but keep any rendered fat in the pan.
          \end{itemize}
        \item Drain the potatoes
          \begin{itemize}[label=\textbullet,leftmargin=1.5em,before=\mdseries]
            \item Once the potatoes are cooked through and tender, strain them. They should be done in about 10-15 minutes, but you can test them with a knife, which should poke through with little resistance.
          \end{itemize}
        \item Make the picadillo sauce base
          \begin{itemize}[label=\textbullet,leftmargin=1.5em,before=\mdseries]
            \item In the same pan, you used to brown the beef (there should be enough fat, but add a bit of oil if necessary) add the onion and garlic mixture and let those sauté for a minute or two until fragrant.
            \item Then add the paprika, chile powder, and cinnamon.
            \item Next, add in ½ can of whole peeled tomatoes and mash them with a potato masher to create a sauce-like consistency.
            \item Once the sauce has formed, add the browned beef back in, and use the potato masher to turn it into small, pebbly bits.
          \end{itemize}
        \item  Finish the picadillo
          \begin{itemize}[label=\textbullet,leftmargin=1.5em,before=\mdseries]
            \item Add the cooked and drained potatoes and mix together before adjusting the final seasoning and texture.
            \item Add salt to taste, and optionally a squirt of hot sauce for acidity and spice. Add more of the other seasonings if desired.
            \item Lastly, adjust the consistency as needed:
            \item To make it more saucy, add a splash of water.
            \item To make it thicker, let it cook down a bit longer. For tostadas, thicker is usually better.
          \end{itemize}
        \item  Assemble \& serve
          \begin{itemize}[label=\textbullet,leftmargin=1.5em,before=\mdseries]
            \item Take a tostada and add a thick layer of warm picadillo as the base.
            \item Top with your choice of crema or sour cream, lettuce, tomatoes, pickled onions, and queso fresco. Enjoy
          \end{itemize}
      \end{itemize}
    \end{minipage}
  }

\newcommand{\cmdSubText}[2]{%
  \ifthenelse{\equal{#2}{}} {} {\cmdSubtitle{#1}{#2}}
}

\sbox{\myTitle}{%
  \begin{minipage}[t]{0.6\textwidth}\raggedright%
    \textbf{\LARGE\bfseries \myTitleText} \vspace{2ex} \\
    \cmdSubText{PREP~TIME}{\myPrepTime}\quad
    \cmdSubText{COOK~TIME}{\myCookTime}\quad
    \cmdSubText{TOTAL~TIME}{\myTotalTime}\quad \\
    \cmdSubText{SERVINGS}{\myServings}\vspace{0.5ex} \\
    \cmdSubText{AUTHOR}{\myAuthor}\\
    \cmdSubText{SOURCE}{\href{\myURLText}{\myTitleText} \vspace{1ex} \\}
    \myPreamble{}
  \end{minipage}
}

\sbox{\myImage}{%
  \begin{minipage}[t]{0.35\textwidth}
    \ifthenelse{\equal{\myFilename}{}}{}{\adjustbox{valign=t}{\includegraphics[width=\textwidth]{\myFilename}}}
  \end{minipage}
}

\framebox{%
  \cmdRecipe{recipelist}{\myIndexEntry}{\myTitleText}
  \cmdRecipe{authorlist}{\myAuthor}{\myTitleText}
\begin{tabular}{@{}llllll@{}}
  \multicolumn{4}{@{}l@{}}{\usebox{\myTitle}} &
  \multicolumn{2}{@{}l@{}}{\usebox{\myImage}} \\
  \multicolumn{6}{@{}l@{}}{} \rule{0.95\linewidth}{0.4pt}\vspace{2ex} \\ 
  \multicolumn{2}{@{}l}{\usebox{\myIngred}} &
  \multicolumn{4}{@{}l}{\usebox{\myMethod}} \\
\end{tabular}
}

\end{document}
