% vim: ft=tex:
\documentclass[../web-recipes.tex]{subfiles}
\begin{document}

\renewcommand{\mytitle}{Oven Ribs (theKitchn)}
\renewcommand{\myurl}{https://www.thekitchn.com/how-to-make-great-ribs-in-the-oven-cooking-lessons-from-the-kitchn-96973}

\sbox{\ingredients}{%
  \begin{minipage}[t]{0.32\textwidth}
    {\sffamily\bfseries Ingredients}\vspace{0.5ex}
    \begin{itemize}
      \item 4 to 5 pounds pork spareribs or baby back ribs
      \item \nicefrac{1}{4} cup Dijon mustard
      \item 1 to 2 teaspoons liquid smoke (optional)
      \item 1 cup spice rub
      \item 1 cup barbecue sauce, plus more for serving
    \end{itemize}
  \end{minipage}
}

\sbox{\method}{%
  \begin{minipage}[t]{0.55\textwidth}
    {\sffamily\bfseries Method}\vspace{0.5ex}
    \begin{description}
      \item [Prepare the baking sheet] Line a rimmed baking sheet with aluminum foil. Fit a wire
        cooling rack on top. Lay the ribs on top of the rack in a single layer. This arrangement
        allows for heat circulation on all sides of the ribs.
      \item [Season the ribs] Stir the mustard and the liquid smoke together, if using, and brush the
        ribs on both sides. Sprinkle the ribs with the dry rub and pat gently to make sure the rub
        adheres to the rib meat. (Note: This step can be done the day ahead for a deeper flavor.
        Wrap the seasoned ribs in plastic wrap and refrigerate.)
      \item [Broil the ribs] Arrange an oven rack a few inches below the heating element and heat the
        broiler. Make sure the meaty side of the ribs is facing up. Broil until the sugar in the dry
        rub is bubbling and the ribs are evenly browned, about 5 minutes.
      \item [Bake the ribs] Set the oven to 300°F. Move the ribs to an oven rack in the middle of the
        oven. Bake 2 \nicefrac{1}{2} to 3 hours for spareribs or 1 \nicefrac{1}{2} to 2 hours for baby back ribs. Halfway
        through cooking, cover the ribs with aluminum foil to protect them from drying out.
      \item [Brush with barbecue sauce] About 30 minutes before the end of cooking, brush the ribs
        with barbecue sauce, re-cover with foil, and continue cooking.
      \item [Rest the ribs and serve] The ribs are done when a knife slides easily into the thickest
        part of the rib meat. Let them rest, covered, for about 10 minutes, and then cut between the
        bones to separate the individual ribs. Serve immediately with extra barbecue sauce for
        dipping.
    \end{description}
  \end{minipage}
}

\myRecipe{Mains!Ribs}{\mytitle{}}

% \addcontentsline{toc}{subsubsection}{\mytitle}
%   \subsubsection*{\mytitle}
  \begin{tabular}{l}
    \usebox{\ingredients}\quad\usebox{\method}\vspace{3ex}\\
    \multicolumn{1}{c}{\small\ttfamily \url{\myurl}} \\
  \end{tabular}
\end{document}
