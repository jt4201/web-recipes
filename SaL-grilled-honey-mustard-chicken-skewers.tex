% arara: testing
% arara: xelatex
% arara: xelatex
% arara: latexmk: { options: ['-c'] }
% vim: tw=86:
\documentclass{newRecipe}
\begin{document}

% FIXME:
\renewcommand{\myTitleText}{Grilled Honey Mustard Chicken Skewers}
\renewcommand{\myURLText}{https://www.saltandlavender.com/grilled-honey-mustard-chicken-skewers/}
\renewcommand{\myFilename}{SaL-grilled-honey-mustard-chicken-skewers}
\renewcommand{\myServings}{4}
\renewcommand{\myAuthor}{Salt \& Lavender}
\renewcommand{\myCookTime}{15 mins}
\renewcommand{\myPrepTime}{10 mins}
\renewcommand{\myTotalTime}{1 hr 25 mins}
\renewcommand{\myPreamble}{
  Succulent chicken skewers with an easy honey mustard marinade.
}

\sbox{\myIngred}{
  \begin{minipage}[t]{\the\dimexpr 0.35\textwidth}
    \textbf{Ingredients}
    \begin{itemize}
      \item 2 large chicken breasts
      \item ½ cup yellow mustard
      \item 3 tablespoons liquid honey
      \item 1 clove garlic minced
      \item Salt \& pepper to taste
    \end{itemize}
  \end{minipage}
}

\sbox{\myMethod}{
  \begin{minipage}[t]{\the\dimexpr 0.6\textwidth - 2em}
    \textbf{Method}
    \begin{enumerate}
      \item  If you're using wooden skewers, it's a good idea to soak them for at
        least half an hour to help prevent splintering and burning.
      \item  Cut the chicken into fairly small pieces and trim off any fat/gristle.
        Place it in a medium-to-large bowl.
      \item  Add the mustard, honey, garlic, and salt \& pepper to a bowl (or just stir
        it right into the measuring cup you used to measure the mustard in). Stir
        until smooth. Pour it over the chicken, and give the chicken a stir so it's
        coated.
      \item  Refrigerate the chicken for at least an hour, preferably 2+ hours. If
        marinating overnight, cover it with plastic wrap.
      \item  Once your chicken is done marinating, oil your BBQ/grill's grate and
        pre-heat it to high. Thread the chicken onto skewers (I had 5 skewers). Pour
        any extra marinade over them.
      \item  Reduce heat to medium-high. Cook the chicken for about 15 minutes total
        time (or until it's cooked through), turning every few minutes. We used our
        gas BBQ, cooking the chicken with the lid down.
    \end{enumerate}
  \end{minipage}
}

\newcommand{\cmdSubText}[2]{%
  \ifthenelse{\equal{#2}{}} {} {\cmdSubtitle{#1}{#2}}
}

\sbox{\myTitle}{
  \begin{minipage}[t]{0.6\textwidth}\raggedright
    \textbf{\LARGE\bfseries \myTitleText} \vspace{2ex} \\
    \cmdSubText{PREP~TIME}{\myPrepTime\quad}
    \cmdSubText{COOK~TIME}{\myCookTime\quad}
    \cmdSubText{MARINADE~TIME}{1 hr\quad \\}
    \cmdSubText{TOTAL~TIME}{\myTotalTime\quad \\}
    \cmdSubText{SERVINGS}{\myServings \vspace{0.5ex} \\}
    \cmdSubText{AUTHOR}{\myAuthor \\}
    \cmdSubText{SOURCE}{\href{\myURLText}{\myTitleText} \vspace{1ex} \\}
    \myPreamble{}
  \end{minipage}
}

\sbox{\myImage}{
  \begin{minipage}[t]{0.35\textwidth}
    \adjustbox{valign=t}{\includegraphics[width=\textwidth]{images/\myFilename}}
  \end{minipage}
}

\framebox{
\begin{tabular}{@{}llllll@{}}
  \multicolumn{4}{@{}l@{}}{\usebox{\myTitle}} &
  \multicolumn{2}{@{}l@{}}{\usebox{\myImage}} \\
  \multicolumn{6}{@{}l@{}}{} \rule{0.95\linewidth}{0.4pt}\vspace{2ex} \\ 
  \multicolumn{2}{@{}l@{}}{\usebox{\myIngred}} &
  \multicolumn{4}{@{}l@{}}{\usebox{\myMethod}} \\
\end{tabular}
}

\end{document}
